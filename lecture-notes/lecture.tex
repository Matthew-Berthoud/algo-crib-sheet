\documentclass{article}

\usepackage{exscale}
\usepackage{graphicx}
\usepackage{amsmath}
\usepackage{latexsym}
\usepackage{times,mathptm}


\oddsidemargin -0.1in
\textwidth 6.8in
\topmargin -0.5in
\textheight 9in
%\def\baselinestretch{1.1}

\begin{document}

\begin{Large}
\centerline {\bf CS653 Analysis of Algorithms}
\end{Large}

\section{Mathematical Foundations}

\subsection{Common functions}

\noindent{\em Reading: CLRS 3.2 }

\begin{itemize}

\item Ceilings and floors: $\lceil x \rceil$ and $\lfloor x \rfloor$.

\item Modular arithmetic: $a\bmod n=a-\lfloor a/n\rfloor n$.
$a\equiv b\bmod n$ iff $a\bmod n= b\bmod n$.

\item Polynomials: $p(n)=\sum_{i=0}^d a_in^i$. (Note: Coefficients
$a_i$ and degree $d$ are constants.)

\item Exponentials: $a^0=1$, $a^{-1}={1\over a}$, $a^m\cdot a^n=a^{m+n}$,
$a^m/a^n=a^{m-n}$.

$e^x=1+x+{x^2\over 2!}+{x^3\over 3!}+\cdots$. (Note: $e=2.71828\ldots$.)

$\lim_{n\to\infty}{(1+{x\over n})}^n=e^x$.

\item Logarithms: $\log n=\log_2 n$ or $\log_c n$ for some $c$ we don't care
about.

$\log(ab)=\log a+\log b$, $\log ({a\over b})=\log a-\log b$.

$\log_a b={{\log_c b}\over {\log_c a}}$.

$\log_a b^n=n\log_a b\not={(\log_a b)}^n$, $a^{\log_a n}=n$,
$a^{\log_c b}=b^{\log_c a}$.

$ln(1+x)=x-{x^2\over 2}+{x^3\over 3}-{x^4\over 4}+\cdots$.

\item Factorials: $n!=n\cdot (n-1)\cdots 2\cdot 1$.

$n!=n\cdot (n-1)!$, $0!=1$.

Sterling's approximation: $n!=\sqrt{2\pi n}({n\over e})^n(1+
\Theta({1\over n}))$. (Note: $\Theta$ means having the same order of 
magnitude.) The following approximation also holds:
$n!=\sqrt{2\pi n}({n\over e})^ne^{\alpha_n}$, where
${1\over 12n+1}<\alpha_n<{1\over 12n}$.

$\log n!=\Theta(n\log n)$.

\item Functional iteration: A function $f$ applied iteratively $i$ times
to an initial argument $n$. Defined recursively, $f^{(0)}(n)=n$ and
$f^{(i)}(n)=f(f^{(i-1)}(n))$ for $i>0$. (Note: The distinction between
$f^{(i)}(n)$ and $f^i(n)$.)

For example, if $f(n)=2n$ then $f^{(i)}(n)=2^in$.

\item The log star function: $\log^* n=\min\{i\ge0:\log^{(i)}n\le1\}$,
which is a very slowly growing function. $\log^*2=1$, $\log^* 4=2$,
$\log^*16=3$, $\log^*65536=4$, $\log^*2^{65536}=5$.

\item Combinatorics: Permutation $n!$ (order matters);
Combination ${n\choose k}={n!\over {k!(n-k)!}}$ (order doesn't matter).

Pascal's triangle: ${n\choose k}={{n-1}\choose k}+{{n-1}\choose {k-1}}$.

Binomial expansion: $(x+y)^n=\sum_{k=0}^n {n\choose k}x^ky^{n-k}$.

\end{itemize}

\subsection{Sums}

\noindent{\em Reading: CLRS A}

\begin{itemize}

\item Arithmetic: $1+2+\cdots+n={1\over2}n(n+1)$.

\item Geometric: $1+r+r^2+\cdots+r^n={{r^{n+1}-1}\over{r-1}}$ for
$r\not=1$.

$1+r+r^2+\cdots={1\over{1-r}}$ for $|r|<1$.

\item Harmonic: $H_n=1+{1\over2}+{1\over3}+\cdots+{1\over n}
=\ln n+\gamma+{\epsilon\over{2n}}$ for $\gamma=0.5772156649\ldots$
(Euler's constant) and $0<\epsilon<1$.

{\em Example:} Prove that $H_n<\ln n+1$.

{\em Proof:} $H_n=\sum_{i=1}^n{1\over i}=1+\sum_{i=2}^n{1\over i}
<1+\int_1^n {1\over x}dx=1+\ln n$.

\vskip 0.25cm
\begin{center}
\includegraphics[height=1.5in]{f1.pdf}
\end{center}

{\em Remark:} Use integrals to bound summations:
Assume $f(x)$ is monotonically decreasing, then
$\int_m^{n+1} f(x)dx\le \sum_{k=m}^nf(k)\le\int_{m-1}^nf(x)dx$.
(What if the function is monotonically increasing?)

\item Binomial: ${n\choose 0}+{n\choose 1}+{n\choose 2}+
\cdots+{n\choose n}=2^n$.

\item Other useful sums:

$\sum_{i=1}^n i^2={1\over6}n(n+1)(2n+1)$.

$\sum_{i=1}^n i^3=(\sum_{i=1}^n i)^2$.

$\sum_{i=1}^n ix^{i-1}={{nx^{n+1}-(n+1)x^n+1}\over{(x-1)^2}}$.

\end{itemize}

\subsection{Asymptotic notation}

\noindent{\em Reading: CLRS 3.1}

\begin{itemize}

\item Used to compare the growth rate or order of magnitude
of increasing functions. ``Asymptotic'' deals with the 
behavior of functions in the limit, for sufficiently large
values of variables.

\item $f(n)=O(g(n))$ if $\exists c, n_0$ such that 
$f(n)\le cg(n)$ for $n\ge n_0$.

\item $f(n)=\Omega(g(n))$ if $\exists c, n_0$ such that
$f(n)\ge cg(n)$ for $n\ge n_0$.

\item $f(n)=\Theta(g(n))$ if $\exists c_1, c_2, n_0$ such that
$c_1g(n)\le f(n)\le c_2g(n)$ for $n\ge n_0$.

\item $f(n)=o(g(n))$ if $\forall c$ $\exists n_0$ such that
$f(n)<cg(n)$ for $n\ge n_0$. 

\item $f(n)=\omega(g(n))$ if $\forall c$ $\exists n_0$ such that 
$f(n)>cg(n)$ for $n\ge n_0$.

\item {\em Remarks:} 
\begin{itemize}
\item In CLRS, the above notation is defined as sets of functions.
For example, $f(n)\in O(g(n))$. But we will simply follow
the mainstream here.
\item $O(\le)$, $\Omega(\ge)$, $\Theta(=)$, $o(<)$, $\omega(>)$.
\item $f(n)=O(g(n))$ iff $g(n)=\Omega(f(n))$, and
$f(n)=o(g(n))$ iff $g(n)=\omega(f(n))$.
\item $f(n)=\Theta(g(n))$ iff $f(n)=O(g(n))$ and $f(n)=\Omega(g(n))$.
\item $f(n)=O(g(n))$ if $f(n)=o(g(n))$, and
$f(n)=\Omega(g(n))$ if $f(n)=\omega(g(n))$.
\item An alternative definition for $f(n)=o(g(n))$ is 
$\lim_{n\to\infty}{f(n)\over g(n)}=0$. Likewise, an alternative definition
for $f(n)=\omega(g(n))$ is $\lim_{n\to\infty}{f(n)\over g(n)}=\infty$.
\item Asymptotic notation ignores constant factors and lower-order terms.
\item {\em Rule of thumb:} constant $\le$ polylogarithmic $\le$ polynomial
$\le$ exponential $\le$ superexponential.

{\em Example:} $1$, $\sqrt{\log n}$, $\ln n$, $(\log n)^2$, $\sqrt n$,
$\sqrt n \log n$, $10n$, $n\log n$, $n^2$, $n^{\log\log n}$,
$2^n$, $n2^n$, $n!$, $2^{2^n}$.
\item Taking logarithms helps: $f(n)=O(g(n))$ iff $\log f(n)=O(\log g(n))$.
\end{itemize}

\end{itemize}

\subsection{Proof techniques}

\begin{itemize}

\item Proving by contradiction: 

The following three statements are logically equivalent:

\begin{enumerate}
\item If $A$ then $B$.
\item If not $B$ then not $A$.
\item If $A$ and not $B$ then not $C$, where $C$ is a proved fact or axiom.
\end{enumerate}

{\em Example:} Use contradiction to prove that 
(a) There are infinitely many prime numbers; (b) There exist two
irrational numbers $x$ and $y$ such that $x^y$ is rational; and
(c) $\sqrt 2$ is irrational.

{\em Proof:} Exercises or references.

\item Proving by induction:

The following statements are mathematically equivalent:

\begin{enumerate}
\item $P(n)$ for integers $n\ge c$.
\item Simple integer induction: $P(c)$ and $P(n-1)\rightarrow P(n)$. 
(What are inductive basis, inductive hypothesis, and inductive step?) 
\item General integer induction: $P(c)$ and $(\forall i:c\le i\le n-1)
P(i)\rightarrow P(n)$.
\end{enumerate}

There are two additional types of induction:

\begin{enumerate}
\item Structural induction: To prove a property $P(X)$ of a recursively
defined structure $X$, it is suffice to prove that $P(X)$ is true for
the basis structure $X$ and $P(Y_1)\land P(Y_2)\land\cdots\land P(Y_k)
\rightarrow P(X)$, where $X$ is constructed from $Y_1, Y_2, \ldots, Y_k$.
\item Mutual induction: Used in the case when a single statement
$P_1(n)$ can not be proved by induction but rather $P_1(n)$ together
with some other statements $P_2(n),\ldots$ can be proved successfully
by induction.
\end{enumerate}

{\em Example:} Use induction to prove that (a) The tiling 
problem can always be solved; (b) Every positive composite
integer can be expressed as a product of prime numbers; and
(c) Every tree has one more node than it has edges.

{\em Proof:} Exercises or references.

{\em Remark:} What is the tiling problem? Given a board divided
into $m=2^k$ squares in each row and each column with one square
marked special. Given a supply of tiles, each being a $2\times 2$
board with one square removed. The puzzle is to cover the board with
the tiles so that each square is covered exactly once, with the
exception of the special square, which is not covered at all. 

\end{itemize}

\subsection{Solving recurrences}

\noindent{\em Reading: CLRS 4}

\begin{itemize}

\item Recurrence is an equation or inequality that defines a function
in terms of the function's values on smaller inputs. For example,
$T(1)=\Theta(1)$ (boundary condition) and $T(n)=2T({n\over 2})+\Theta(n)$ for 
$n\ge 2$ or almost equivalently, $T(1)=1$ and $T(n)=2T({n\over2})+n$ for $n\ge2$.

\item {\em Remark:} We may neglect some technical details due to our
interest in asymptotic solutions: 
\begin{itemize}
\item Relax the integer argument requirement
on functions.
\item Assume boundary condition $T(n)=\Theta(1)$ for small $n$
if not defined.
\item Use $\Theta(f(n))$ or $f(n)$ at will in the recursive definition.
\end{itemize}

\item The substitution method: Guess and verify.

{\em Example:} $T(n)=2T({n\over2})+n$.

First guess $T(n)=O(n\log n)$. Next prove that 
$T(n)\le cn\log n$ for some $c$ and $n\ge 2$. (Why not start from $n=1$?)

Inductive basis: For $n=2$, $T(2)=2T(1)+2=4\le c2\log 2$ if $c\ge2$.

Inductive step: Assume $T({n\over2})\le c{n\over2}\log{n\over2}$.
\begin{eqnarray*}
T(n)&=&2T({n\over2})+n\\
&\le&cn\log{n\over2}+n\\
&=&cn\log n-cn+n\\
&<&cn\log n.
\end{eqnarray*}

{\em Remarks:}
\begin{itemize}
\item Making a good guess. 
\item To prove $T(n)=O(f(n))$, sometimes we use an inequality
stronger than $T(n)\le cf(n)$ in the induction (such as 
$T(n)\le 20cf(n)$).

\item Avoid using asymptotic notation in the inductive proof.

{\em Exercise:} $T(n)=T(n-1)+n$. What is wrong with the following proof?

First guess $T(n)=O(n)$.

Inductive basis: For $n=1$, $T(1)=1=O(1)$.

Inductive step: Assume $T(n-1)=O(n-1)$
\begin{eqnarray*}
T(n)&=&T(n-1)+n\\
&=&O(n-1)+n\\
&=&O(n).
\end{eqnarray*}
\item Changing variables.

{\em Example:} $T(n)=2T(\sqrt{n})+\log n$. 

First, renaming $m=\log n$ yields $T(2^m)=2T(2^{m/2})+m$.
Next, renaming $S(m)=T(2^m)$ yields $S(m)=2S(m/2)+m$. 
Since $S(m)=O(m\log m)$, $T(n)=T(2^m)=S(m)=O(m\log m)=O(\log n\log\log n)$.
\end{itemize}

\item The iteration method: Apply recurrence until a summation pattern 
can be figured out.

{\em Example:} $T(n)=3T({n\over4})+n$.

Assume $n=4^k$, so $k=\log_4 n$.
\begin{eqnarray*}
T(n)&=&3T({n\over4})+n\\
&=&3^2T({n\over{4^2}})+{3\over4}n+n\\
&=&3^3T({n\over{4^3}})+({3\over4})^2n+({3\over4})n+n\\
&=&\cdots\\
&=&3^kT({n\over{4^k}})+(({3\over4})^{k-1}+\cdots+({3\over4})+1)n\\
&=&3^{\log_4 n}+{{1-({3\over4})^{\log_4 n}}\over{1-{3\over4}}}n\\
&=&4n-3\cdot 3^{\log_4 n}\\
&=&O(n).
\end{eqnarray*}

{\em Exercise:} Solve $T(n)=\sqrt nT(\sqrt n)+n$ by iteration.

\item The recursion-tree method: 
Similar to the iteration method, use a tree for bookkeeping.

{\em Example:} $T(n)=T({n\over3})+T({2n\over3})+n$.

\vskip 0.25cm
\begin{center}
\includegraphics[height=2.75in]{f2.pdf}
\end{center}

{\em Exercise:} Solve $T(n)=T(\alpha n)+T((1-\alpha)n)+n$, where
$0<\alpha<1$, by recursion tree.

\item The master method:

{\em Theorem:} If $T(n)=aT({n\over b})+f(n)$ for $a\ge1$ and $b>1$, then

(a) if $f(n)=O(n^{(\log_ba)-\epsilon})$ for some $\epsilon>0$, then
$T(n)=\Theta(n^{\log_ba})$;

(b) if $f(n)=\Theta(n^{\log_ba})$, then
$T(n)=\Theta(n^{\log_ba}\log n)$;

(c) if $f(n)=\Omega(n^{(\log_ba)+\epsilon})$ for $\epsilon>0$ and if
$af({n\over b})\le cf(n)$ for $c<1$ and all large $n$, then
$T(n)=\Theta(f(n))$.

{\em Remark:} The master method does not cover all cases.

{\em Example:} $T(n)=3T({n\over4})+n\log n$.

$a=3$, $b=4$, and $f(n)=n\log n$.
Case (c) applies. So $T(n)=\Theta(n\log n)$.

{\em Exercise:} Solve $T(n)=4T({n\over2})+f(n)$ by the master 
theorem for $f(n)=n, n^2, n^3$.

\end{itemize}

\newpage

\section{Analysis of Algorithms}

\subsection{An overview}

\noindent{\em Reading: CLRS 1}

\begin{itemize}

\item Problem types:

%By requirement: Search, structuring, construction, optimization,
%decision, adaptive.

By difficulty: conceptually hard, analytically hard,
computationally hard, computationally unsolvable.

\item Algorithms:

Correctness, finiteness, definiteness (unambiguity), effectiveness, output.

Pseudocode = human language + math language + computer language.

Analysis of algorithms: w/o programming, testing, and debugging.

\item Complexity measures: Time, space $\leftrightarrow$
worst, average.

Time is measured as the number of basic steps in a prespecified
computation model.

\item Models of computation (to define basic steps):

Turing machine: Basic step: Read-write-move-change.

Random Access Machine: Basic step: LOAD, STORE, ADD, SUB, MULT, DIV,
READ, WRITE, JUMP, JGTZ, JZERO, HALT.

Pseudocode: Basic step: $+$, $-$, $*$, $/$, assignment, compare, if,
i/o. (for, while, $**$ are not.)

Polynomial equivalence of all reasonable computing models.

\item Instance: A specific input.

{\em Example:} Problem: Sort a list of numbers into increasing order;
Instance: $L=(5,1,4,2,3)$.

\item Instance (Input) size: 
The amount of space used to store the instance.

{\em Example:}
\begin{center}
\begin{tabular}{r|l}
Instance & Size\\\hline
List & \# of items \\
Matrix & \# of rows and columns \\
Graph & \# of vertices and edges \\
Integer & \# of bits
\end{tabular}
\end{center}

\item Time complexity: $T(n)$: \# of basic steps in the pseudocode model.

{\em Remarks:}
\begin{itemize}
\item Functions of input size, not input value, represented in $O$, $\Omega$,
$\Theta$ (easy to compare; asymptotic performance for large input size).
\item Input-value independent, programming independent, language independent,
machine independent.
\end{itemize}

\item A spectrum of complexity:

Classify problems into categories according to the worst-case time
complexity of their fastest possible (optimal) algorithms.

\begin{center}
\begin{tabular}{r|l}
Degree of difficulty & Problem examples\\\hline\hline
Unsolvable & Halting problem\\\hline
Superexponential & Presburger arithmetic $2^{2^n}$\\
Exponential & Circularity attribute grammar $2^n$\\\hline
Polynomial & Linear programming (LP)\\
$n^c$, for $2\le c<3$ & Matrix multiplication\\
$n\log n$ & Sorting by comparison\\
$n$ & Selecting the $k$th largest
\end{tabular}
\end{center}

\end{itemize}

\subsection{Worst-case analysis}

\noindent{\em Reading: CLRS 2.1 and 2.2}

\begin{itemize}

\item Definition: 

$I_n$: Any instance of size $n$;

$t(I_n)$: Time (\# of basic steps) spent on $I_n$ by the algorithm;

$T(n)$: Worst-case time complexity of any instance of size $n$,
$T(n)=\max_{\forall I_n}\{t(I_n)\}$.

\item Insertion sort: $A\Rightarrow A$ sorted in increasing order.

Insertion\_Sort($A$)
\par\quad for $j\leftarrow 2$ to $n$
\par\quad\quad $key\leftarrow A[j]$
\par\quad\quad $i\leftarrow j-1$
\par\quad\quad while $i>0$ and $A[i]>key$
\par\quad\quad\quad $A[i+1]\leftarrow A[i]$
\par\quad\quad\quad $i\leftarrow i-1$
\par\quad\quad $A[i+1]\leftarrow key$

To insert $A[j]$ into the sorted $A[1\ldots j-1]$, the algorithm
will make at most $j-1$ comparisons and $j-1$ shifts. So the overall 
time complexity,
which is dominated by the number of comparisons and the number of
shifts, is at most $2\sum_{j=2}^n(j-1)
=\Theta(n^2)$.

\item Euclid's algorithm (300 B.C.): Greatest common divisor.

If $n\not=0$, then $gcd(m,n)=gcd(n, m \bmod n)$;

If $n=0$, then $gcd(m,n)=m$.

Euclid($m,n$)\qquad //Assume $m\ge n$
\par\quad while $n>0$
\par\quad\quad $t\leftarrow m\bmod n$
\par\quad\quad $m\leftarrow n$
\par\quad\quad $n\leftarrow t$
\par\quad return $m$

The time complexity of the algorithm is determined by the number of iterations
of the while-loop. Let it be $k$. Also let $m_i$ and $n_i$ be the values of
$m$ and $n$ at the end of the $i$th iteration, respectively.

We observe that (1) $n_k=0$, $n_i\ge 1$ for $i<k$, and 
$n_0>n_1>\cdots>n_k$; (2) $m_i=n_{i-1}$, which implies that $m_i\ge n_i$ 
for $i\ge1$; (3) $n_i=m_{i-1} \bmod n_{i-1}<{m_{i-1}\over2}={n_{i-2}\over2}$
for $i\ge2$. (Note: If $a\ge b$, then $a\bmod b<{a\over2}$.)

We then have $1\le n_{k-1}<{n_{k-3}\over2}<{n_{k-5}\over2^2}<\cdots<
{(n_1,n_0)\over2^{\lceil k/2\rceil-1}}\le {n_0\over2^{\lceil k/2\rceil-1}}$.

Thus, ${k\over2}-1\le\lceil{k\over2}\rceil-1<\log n_0$,
which implies $k\le 2+2\log n_0$.

So $T(m,n)=O(k)=O(\log n)$.

\item Multiplying two integers $x$ and $y$: 

Assume that $x$ (multiplicand) and $y$ (multiplier) 
have the same number of figures, $n=2^k$,
denoted $(x)_n$ and $(y)_n$.
Then $(x)_n=10^{n\over2}\cdot(a)_{n\over2}+(b)_{n\over2}$ and
$(y)_n=10^{n\over2}\cdot(c)_{n\over2}+(d)_{n\over2}$.

So $(x)_n\cdot(y)_n=10^n((a)_{n\over2}\cdot (c)_{n\over2})+
10^{n\over2}((a)_{n\over2}\cdot (d)_{n\over2}+ (b)_{n\over2}\cdot
(c)_{n\over2})+(b)_{n\over2}\cdot (d)_{n\over2}$. The problem of 
multiplying two integers of size $n$ is now reduced to four subproblems
of multiplying two integers of size $n\over2$.

$T(n)=4T({n\over2})+n=\Theta(n^2)$ by the master theorem.

Now let $s=(a)_{n\over2}\cdot (c)_{n\over2}$,
$t=(b)_{n\over2}\cdot (d)_{n\over2}$, and
$r=((a)_{n\over2}+(b)_{n\over2})\cdot((c)_{n\over2}+(d)_{n\over2})$.
Then $(x)_n\cdot(y)_n=10^ns+10^{n\over2}(r-s-t)+t$. Only three
subproblems have to be solved in the recursion.

$T(n)=3T({n\over2})+n=\Theta(n^{\log_23})=\Theta(n^{1.585})$ 
by the master theorem.
\end{itemize}

%\newpage
\subsection{Average-case analysis}

\noindent{\em Reading: CLRS 2.2}

\begin{itemize}

\item Definition:

$I_n$: Any instance of size $n$;

$t(I_n)$: Time (\# of basic steps) spent on $I_n$ by the algorithm;

$p(I_n)$: Probability that $I_n$ appears as input;

$T(n)$: Average-case time complexity of any instance of size $n$,
$T(n)=\sum_{\forall I_n}p(I_n)t(I_n)$.

\item Insertion sort:

Assumptions: (1) Distinct items; (2) Each permutation with equal
probability to occur.

$T(n)=$ average \# of pairwise comparisons $=\sum_{j=2}^n c_j$, where
$c_j$ is the average \# of comparisons to insert $A[j]$ ($key$) into the sorted
$A[1\ldots j-1]$. For $key$, there are $j$ possible positions, with 
probability ${1\over j}$ for each.

So $c_j={1\over j}\cdot 1+{1\over j}\cdot 2+\cdots+{1\over j}\cdot (j-1)
+{1\over j}\cdot (j-1)={1\over2}(j+1)-{1\over j}$.

Therefore, $T(n)=\sum_{j=2}^n({1\over2}(j+1)-{1\over j})
={1\over4}n^2+{3\over4}n-H_n=\Theta(n^2)$.

\item Binary search

Given $A[1\ldots n]$ sorted and $x$ (a query). Is $x$ a member of $A$?

{\em Example:} Let $n=7$. The binary search algorithm can be illustrated
by the following decision tree.

\vskip 0.25cm
\begin{center}
\includegraphics[height=1.1in]{f3.pdf}
\end{center}

Assumptions: (1) $x\in A$; (2) Distinct items in $A$; (3)
$Pr(x=A[i])={1\over n}$ for any $i=1,2,\ldots,n$; (4) $n=2^k-1$ for some $k$.

Binary search $\Rightarrow$ Decision tree of $n$ nodes (full binary tree
with $k$ levels).

Average-case time $\Rightarrow$ Average length of a path from the root to
a tree node.

\begin{center}
\begin{tabular}{l|l|l}
Level & \# of nodes & average length\\ \hline
1 & 1 & ${1\over n}\cdot 1$\\
2 & 2 & ${2\over n}\cdot 2$\\
3 & 4 & ${4\over n}\cdot 3$\\
$\cdots$ & $\cdots$ & $\cdots$\\
$k$ & $2^{k-1}$ & ${2^{k-1}\over n}\cdot k$
\end{tabular}
\end{center}

\begin{eqnarray*}
T(n)&=&\sum_{i=1}^k {2^{i-1}\over n}\cdot i\\
&=&{1\over n}\sum_{i=1}^k i\cdot 2^{i-1}\\
&=&{1\over n}(k\cdot 2^k-2^k+1)\\
&=&k+{k\over n}-1\\
&=&\Theta(\log n)
\end{eqnarray*}

\item Binary search trees (BST)

What is the average number of comparisons needed to insert $n$
distinct random elements into an initially empty BST?

$T(0)=T(1)=0$.

Assume that $A=(a_1,\ldots,a_n)$ is the list given and that 
$B=(b_1,\ldots,b_n)$ is $A$ sorted increasingly.
That $A$ is a random sequence implies that $a_1$ is equally likely
to be $b_j$ for any $1\le j\le n$. Consider the tree obtained after
the insertion of all $n$ numbers.

\vskip 0.25cm
\begin{center}
\includegraphics[height=1.1in]{f4.pdf}
\end{center}

So $T(n)={1\over n}\sum_{j=1}^n(n-1+T(j-1)+T(n-j))=n-1+{2\over n}
\sum_{j=0}^{n-1}T(j)$ for $n\ge2$.

To solve this recurrence, we guess $T(n)=O(n\log n)$, i.e., 
$T(n)\le cn\ln n$ for some $c$. We prove by induction that
this is indeed the case.

When $n=1$, $T(n)=0\le c\cdot 1\cdot \ln 1$ for any $c$. So the
basis holds. Assume that when $j\le n-1$, $T(j)\le cj\ln j$.
Now consider $n$.
\begin{eqnarray*}
T(n)&=&n-1+{2\over n}\sum_{j=0}^{n-1}T(j)\\
&\le&n-1+{2\over n}\sum_{j=1}^{n-1}cj\ln j\\
&\le&n-1+{2\over n}c\int_2^n x\ln xdx\\
&\le&n-1+{2\over n}c({1\over2}n^2\ln n-{1\over4}n^2)\\
&=&n-1+cn\ln n-{c\over2}n\\
&\le&cn\ln n {\rm \quad for\ \ } c\ge  2
\end{eqnarray*}
\end{itemize}

\newpage
\subsection{Amortized analysis}

\noindent{\em Reading: CLRS 17.1--17.3}

\begin{itemize}

\item Amortization:

(1) To put money aside at intervals, as a sinking fund, for gradual payment
of a debt;

(2) To average the running times of operations in a sequence over the
entire sequence.

\item Motivation:

A sequence of data structure related operations, rather than just a single
operation, is performed. (An operation may change the data structure, thus
affect the next operation.) 

When is the total time complexity of the entire sequence?

Worst-case analysis: sum of worst-case time of each operation (which may
never be achieved).

Average-case analysis: average over all possible inputs, 
inaccurate probabilistic assumptions, hard.

Amortized analysis: average over successive operations,
average the time per operation over a worst-case sequence.
It guarantees the average performance of each operation in the worst case.

\item Stack manipulation: an example.

Unit-time primitives: push and pop.

An operation contains zero or more pop followed by a push.

A sequence contains $m$ operations defined above. For instance,
(push), (pop, push), (push), .... Assume that any given sequence
never pops when the stack is empty.

Initially the stack is empty.

Question: What is the total time/cost of a sequence of $m$ operations?
(as a function of $m$)

Worst-case analysis: $m^2$

Amortized analysis: $2m$ (Each operation causes one push and possibly
one later pop.)

\item A banker's view of amortization:

To represent prepaid work as credit stored with specific
objects within the data structure.

Coin-operated computer: 1 credit $\Rightarrow$ 1 time unit
$\Rightarrow$ 1 primitive.

Sequence of operations: $O_1, \ldots, O_m$, where operation $O_i$
is allocated with credits $c_i$.

Assumptions: (1) Unused credits are carried over to later operations;
(2) Operations can borrow credits as long as any debt is paid off
eventually.

If all sequences of length $m$ can be performed with the allocated credits,
then the total time is no larger than $\sum_{i=1}^m c_i$. 

Keys: (1) How to pick the smallest possible $c_i$; (2) How to prove
all sequences can be performed with the allocated credits $c_1,\ldots,c_m$.

Consider the stack manipulation example. Let $c_i=2$ for $i=1,\ldots,m$.
For any operation, one credit is used to execute push in the operation, 
while the other is saved for pop (of the same element) in a later
operation. So in other words, each push is paid right away and each
pop is paid for by a saved credit. So the total time is bounded
from above by $\sum_{i=1}^m c_i=2m$.

\item A physicist's view of amortization:

To represent prepaid work as potential energy that can be released for future
operations. The potential is associated with the data structure as a whole
rather than with specific objects within the data structure.

Define a potential function $\Phi: D\rightarrow R$. Assume that
an operation takes $t_i$ time units to change the data structure
from $D_{i-1}$ to $D_i$. Then the amortized time of the operation is
$$a_i=t_i+\Phi(D_i)-\Phi(D_{i-1}),$$ and the total time for the 
sequence is 
\begin{eqnarray*}
\sum_{i=1}^mt_i&=&\sum_{i=1}^m(a_i-\Phi(D_i)+\Phi(D_{i-1}))\\
&=&\Phi(D_0)-\Phi(D_m)+\sum_{i=1}^ma_i\\
&\le&\sum_{i=1}^ma_i{\rm \quad if\ \ }\Phi(D_0)\le\Phi(D_m).
\end{eqnarray*}
\end{itemize}

Key: How to choose $\Phi$.

Consider the stack manipulation example. The underlining data 
structure is the stack. Define $\Phi(D)=$\# of items in the 
stack. Since $\Phi(D_0)=0$ and $\Phi(D_m)\ge0$, then
$\Phi(D_0)\le\Phi(D_m)$.
If $O_i$ has $k$ pops and $1$ push, and $D$ contains $j$ items 
before $O_i$ is performed, then $\Phi(D_{i-1})=j$ and 
$\Phi(D_i)=j-k+1$. So $a_i=t_i+\Phi(D_i)-\Phi(D_{i-1})=
(k+1)+(j-k+1)-j=2$. Therefore, the total time is
$\sum_{i=1}^mt_i\le\sum_{i=1}^ma_i=2m$.

\newpage
\subsection{Disjoint sets}

\noindent{\em Reading: CLRS 21}

\begin{itemize}

\item Purposes: (1) Use good data structure to achieve algorithm efficiency;
(2) Use amortization (it's accounting method) to give tight estimation of 
algorithm complexity.

\item Problem: Initially, we are given $n$ singletons: 
$S_i=\{i\}$ for $i=1,\ldots,n$. We wish to execute a sequence of
operations of the following two types: (1) $union(S_i,S_j)$ returns
$S_i\cup S_j$, where $S_i$ and $S_j$ are disjoint; (2) $find(i)$
returns the name of the set containing $i$. How can we organize
the data (sets) such that any sequence of intermixed unions and
finds can be performed efficiently?

\item Data structure: Set $\Rightarrow$ tree (arbitrary); 
Set name $\Rightarrow$ root; Sets $\Rightarrow$ forest
(use parent array).

Initially, there are $n$ singletons, corresponding to $n$
single-node trees in a forest, which is represented by a
parent array of size $n$ with $0$ in each entry.
($parent[i]$ gives the parent of $i$ in the forest.
If $parent[i]=0$, $i$ is a root.)

Implementation of union (by size): $union(i,j)$ in $O(1)$.

\vskip 0.25cm
\begin{center}
\includegraphics[height=1.3in]{f5.pdf}
\end{center}

Implementation of find (by path compression): $find(i)$ in $O(d)$,
where $d$ is the depth of $i$ in the tree.

\vskip 0.25cm
\begin{center}
\includegraphics[height=1.3in]{f6.pdf}
\end{center}

\item Complexity: Given an intermixed sequence $\sigma$ of $q$ unions
and $p$ finds. What is the worst-case time complexity for executing
$\sigma$ on $\{1\}, \ldots, \{n\}$?

Note: The worst-case analysis gives an overly pessimistic bound of
$O(q+pn)$ or $O(q+p\log q)$ (by P3 later). We will see next how a tighter bound can be obtained
by an amortized analysis.

Let $\sigma'$ be $\sigma$ with all finds removed. Let $Forest(\sigma')$
be the forest resulting from executing $\sigma'$.

Rank $r(i)$: Height (\# of nodes on the longest path from $i$ to 
a leaf) of the subtree rooted at $i$ in the forest resulting from
executing $\sigma'$.

Properties:

P1: Any tree has no larger than $q+1$ nodes in $Forest(\sigma')$. (Branches).

P2: Node $i$ has at least $2^{r(i)-1}$ descendants (including $i$)
in $Forest(\sigma')$. (Induct on $r(i)$. Consider the subtree rooted
at node $i$, which must be constructed by some union operations. There
must be a union operation that 
merges a smaller-size tree with height $r(i)-1$
to a tree rooted at $i$. Thus the number of nodes in the tree
obtained after the union is at least $2\cdot 2^{r(i)-2}=2^{r(i)-1}$.)

P3: $r(i)\le \lfloor\log (q+1)\rfloor+1$, for any $i$. 
(By contradiction, assume $r(i)\ge \lfloor\log (q+1)\rfloor+2$.
By P2, the number of descendant is at least $2^{\lfloor\log (q+1)\rfloor+1}
>2^{\log (q+1)}=q+1$. A contradiction to P1.)

P4: There are at most ${n\over 2^{r-1}}$ nodes of rank $r$. 
(By contradiction, assume for some rank $r$ there are more than 
${n\over 2^{r-1}}$ nodes with rank $r$. By P2 each node has at least $2^{r-1}$
descendants. The number of nodes in these (disjoint) subtrees is more than
${n\over 2^{r-1}}\cdot 2^{r-1}=n$. A contradiction!)

P5: At any time during the execution of $\sigma$, the ranks of the nodes
on a path from a leaf to a root increase strict monotonically.
(For nodes $w$ and $v$, if $w$ is a proper descendant of $v$ at some time of executing $\sigma$,
then they must still have the same descendant-ancester relation in $Forest(\sigma')$. So $r(w)<r(v)$.)

{\em Definitions:} $F(0)=1$ and $F(n)=2^{F(n-1)}$ for $n\ge1$.
Let $G$ be the inverse of $F$, i.e., $G(n)=\min\{k\ge0|F(k)\ge n\}$.
Note that $G(F(n))=n$ and $G(n)=\log^* n$.

\begin{center}
\begin{tabular}{c|c|c}
$n$ & $F(n)$ & $G(n)$\\\hline
$0$ & $1$ & $0$\\
$1$ & $2$ & $0$\\\hline
$2$ & $4$ & $1$\\\hline
$3$ & $16$ & $2$\\
$4$ & $65536$ & $2$\\\hline
$5$ & $2^{65536}$ & $3$\\
$\cdots$ & $\cdots$ & $\cdots$\\
$16$ & $\cdots$ & $3$\\\hline
$17$ & $\cdots$ & $4$\\
$\cdots$ & $\cdots$ & $\cdots$\\
\end{tabular}
\end{center}

We next partition the ranks into groups by putting rank $r$ into group
$G(r)$. Thus, group $0$ contains rank $1$; group $1$ contains
just rank $2$; group $2$ contains ranks $3$ and $4$; and finally
group $G(\lfloor\log(q+1)\rfloor+1)$ contains the largest possible 
rank, $\lfloor\log(q+1)\rfloor+1$, according to P3. See the table below.

\begin{center}
\begin{tabular}{c|l}
Group & Ranks in the group\\\hline\hline
$0$ & $1$\\\hline
$1$ & $2$\\\hline
$2$ & $3,4$\\\hline
$3$ & $5,\ldots,16$\\\hline
$\cdots$ & $\cdots$\\\hline
$G(\lfloor\log(q+1)\rfloor+1)$ & $\ldots,\lfloor\log(q+1)\rfloor+1$\\
\hline\hline
$g$ & $F(g-1)+1,\ldots,F(g)$\\
\end{tabular}
\end{center}

What is the total cost of $p$ finds? For $find(i)$, the cost of the
operation is the depth of $i$, i.e., \# of nodes from $i$ to the root.
We apportion this cost between the find operation itself and certain
nodes along the path from $i$ to the root using the following rules.
Let $j$ be any node on the path from $i$ to the root.

R1: If $j$ is the root, or $parent[j]$ is the root, or $parent[j]$
is in a different rank group from $j$, assign $1$ unit of cost to the
find operation.

R2: If $j$ and $parent[j]$ are in the same rank group and
$parent[j]$ is not the root, assign $1$ unit of cost to $j$.

The figure below contains an example of apportioning cost.

\vskip 0.25cm
\begin{center}
\includegraphics[height=2in]{f7.pdf}
\end{center}

The total cost incurred by $p$ finds is the total cost assigned to
$p$ finds by R1, denoted $cost(R1)$, plus the total cost assigned to all 
nodes by R2, denoted $cost(R2)$.

\begin{eqnarray*}
cost(R1)&\le&p\cdot({\rm max.\ cost\ assigned\ to\ any\ find})\\
&\le&p\cdot({\rm \#\ of\ different\ rank\ groups}+1)\\
&\le&p\cdot(G(\lfloor\log(q+1)\rfloor+1)+1+1)\\
&\le&O(pG(\log q)).
\end{eqnarray*}

\begin{eqnarray*}
cost(R2)&=&\sum_{\forall g}{\rm total\ cost\ assigned\ to\ nodes
\ by\ R2\ in\ group\ }g\\
&\le&\sum_{\forall g}({\rm \#\ of\ nodes\ in\ group\ }g)\cdot
({\rm max.\ cost\ assigned\ to\ any\ node\ by\ R2\ in\ group\ }g)\\
&\le&\sum_{g=0}^{G(\lfloor\log(q+1)\rfloor+1)}N(g)\cdot C_{max}(g)\\
&\le&\sum_{g=0}^{G(\lfloor\log(q+1)\rfloor+1)}{2n\over F(g)}\cdot F(g)\\
&=&2n(G(\lfloor\log(q+1)\rfloor+1)+1)\\
&=&O(nG(\log q)).
\end{eqnarray*}

What is $N(g)$, \# of nodes in rank group $g$?

\begin{eqnarray*}
N(g)&\le&\sum_{r=F(g-1)+1}^{F(g)}{n\over 2^{r-1}}\qquad{\rm (By\ P4)}\\
&\le&{n\over 2^{F(g-1)}}(1+{1\over2}+{1\over4}+\cdots)\\
&\le&{2n\over F(g)}.
\end{eqnarray*}

What is $C_{max}(g)$, max. cost assigned to any node by R2 in rank group $g$?

Assume for node $i$, its $r(i)$ is in group $g$. So $F(g-1)+1\le r(i)\le F(g)$.
Recall that there are $F(g)-F(g-1)$ ranks in group $g$.
Every time $i$ is assigned $1$ unit of cost by R2, $i$ is 
reconnected to a new parent with a higher rank than its previous parent
(by P5). Node $i$ will be assigned and moved at most $F(g)-F(g-1)$
times until it is connected to a parent in a different rank group 
or to the root. So $C_{max}(g)\le F(g)-F(g-1)\le F(g)$.

The total cost for executing $\sigma$ is the sum of 
(1) $O(q)$ for $q$ unions; and
(2) $O(pG(\log q))+O(nG(\log q))$ for $p$ finds.
So the total is $O(q+pG(\log q)+nG(\log q))$, almost a linear
function since $G(\cdot)$ is close to a constant function.

\end{itemize}

\newpage

\section{Greedy Algorithms}

\subsection{Introduction}

\noindent{\em Reading: CLRS 16.2}

\begin{itemize}
\item Making change: How to pay $x$ to a customer using the smallest
possible number of coins, given the coinage, $c_1<c_2<\cdots<c_n$,
and a unlimited supply of coins of each denomination?

A greedy algorithm: Starting with nothing, at every stage we add to
the coins already chosen a coin of the largest value available that does 
not take us past the amount to be paid. 

Consider the American coinage. If $x=94$, then the change includes
$3$ quarters, $1$ dime, and $1$ nickel and $4$ pennies. A proof, though
surprisingly hard, exists that the proposed greedy algorithm always
gives the optimal (minimum) solution for the American coinage.

The algorithm may not be optimal for other coinages. For example,
let $c_1=1$, $c_2=9$, and $c_3=15$. Assume $x=19$. The greedy algorithm
gives a solution with $5$ coins. However, the optimal solution has
only $3$ coins.

\item Greedy algorithms are suitable for solving optimization
problems with solutions consisting of a sequence of choices.

Greedy algorithms makes the choice that looks the best at the
moment (short-sighted), and never changes the choices made
(stubborn). So they may not always give optimal solutions.

$Greedy$($C$: set) //$C$ is the set of all choices
\par\qquad $S\leftarrow\emptyset$ //$S$ is the solution set
\par\qquad while $S$ is not a solution and $C\not=\emptyset$
\par\qquad\qquad $x\leftarrow$ an element in $C$ that optimizes $Select(x)$
//You pick $Select(x)$ in advance
\par\qquad\qquad $C\leftarrow C-\{x\}$
\par\qquad\qquad if $S\cup\{x\}$ is a feasible solution
then $S\leftarrow S\cup\{x\}$
\par\qquad if $S$ is a solution then return S
\par\qquad else return ``No solution found''

\item Examples of greedy algorithms:

Kruskal's algorithm for finding the minimum spanning tree of an
undirected edge-weighted graph.

Dijkstra's algorithm for finding the shortest paths from a source
node to all other nodes in a directed, edge-weighted graph.

Heuristics for the knapsack problem.

Heuristics for the bin packing problem.

\end{itemize}

\subsection{An activity-selection problem}

\noindent{\em Reading: CLRS 16.1}

\begin{itemize}

\item Activity $i$ has starting time $s_i$ and finishing time $f_i$,
with $s_i<f_i$. We say activities $i$ and $j$ are compatible
if $f_i\le s_j$ or $f_j\le s_i$. Given $S=\{1,2,\ldots,n\}$,
a set of proposed activities that wish to use a resource $R$
which can only be used by one activity at a time. Determine
$S'\subseteq S$ with the maximum size such that any two
activities in $S'$ are compatible.

\item Greedy algorithm 1: Shortest activity first (SAF).

An instance that shows that SAF is not optimal.

\begin{center}
\begin{tabular}{c|cc|c}
$i$ & $s_i$ & $f_i$ & $f_i-s_i$\\\hline
$1$ & $0$ & $3$ & $3$\\
$2$ & $3$ & $6$ & $3$\\
$3$ & $2$ & $4$ & $2$\\
\end{tabular}
\end{center}

\item Greedy algorithm 2: Earliest starting activity first (ESAF).

An instance that shows that ESAF is not optimal.

\begin{center}
\begin{tabular}{c|cc}
$i$ & $s_i$ & $f_i$\\\hline
$1$ & $1$ & $3$\\
$2$ & $0$ & $5$\\
$3$ & $4$ & $6$\\
\end{tabular}
\end{center}

\item Greedy algorithm 3: Earliest finishing activity first (EFAF).

{\em Theorem:} EFAF is optimal.

{\em Proof} When multiple optimal solutions exist for some inputs,
we may use "prove by transformation" to prove optimality of a greedy
algorithm. 
Take any optimal schedule, assume that it is different from the greedy
schedule.
We wish to prove that the OPT schedule can be transformed
to an EFAF schedule of the same number of activities. We use
the following procedure to do the transformation.

\qquad $S\leftarrow\{1,2,\ldots,n\}$
\par\qquad $OPT\leftarrow\{a_1,a_2,\ldots,a_l\}$
\par\qquad for $i\leftarrow 1$ to $l$
\par\qquad\qquad if $\exists j\in S-OPT$ such that $f_j=\min_{k\in S-OPT}
\{f_k\}$, $f_j<f_{a_i}$, and $s_j\ge f_p$, 
\par\qquad\qquad\qquad where $p$ is the immediate
predecessor of $a_i$ in the current $OPT$
\par\qquad\qquad then replace $a_i$ with $j$ in $OPT$, i.e.,
$OPT\leftarrow OPT-\{a_i\}\cup\{j\}$

By applying the above procedure to the OPT schedule, the
resulting schedule is an EFAF schedule, with the same number
of activities as in the OPT schedule. So EFAF is optimal.

Note: When the optimal solution for any input is unique,
we may use "prove by contradiction" to prove optimality.
Again, assume that the optimal solution is different from the
greedy solution. Then show that by making some change to the optimal
solution there is a better solution than the optimal. Thus, a contradiction.

\end{itemize}

\subsection{Scheduling with deadlines}

\noindent{\em Reading: CLRS 16.5}

\begin{itemize}

\item Given a single machine and $n$ unit-time jobs, $J_1,\ldots,J_n$.
Assume that $J_j$ has deadline $d_j$ (integer) and penalty $p_j$
(integer) which occurs when the deadline is missed.
How can we schedule the jobs on the machine so that the total 
penalty is minimized?

\item A greedy algorithm: Schedule jobs with the largest penalty in the
latest possible time slot.

For the following instance, the greedy algorithm defines a schedule
$J_4,J_2,J_3,J_1,J_7,J_6,J_5$ with total penalty of $50$.

\begin{center}
\begin{tabular}{c|ccccccc}
$J_j$ & $J_1$ & $J_2$ & $J_3$ & $J_4$ & $J_5$ & $J_6$ & $J_7$\\\hline
$d_j$ & $4$ & $2$ & $4$ & $3$ & $1$ & $4$ & $6$\\
$p_j$ & $70$ & $60$ & $50$ & $40$ & $30$ & $20$ & $10$\\
\end{tabular}
\end{center}

The following pseudocode describes the algorithm:

\qquad WLOG, assume $p_1\ge p_2\ge\cdots\ge p_n$.
\par\qquad $P\leftarrow 0$ //total penalty so far
\par\qquad for $i\leftarrow 1$ to $n$ $S[i]\leftarrow 0$ //S is the schedule
\par\qquad for $j\leftarrow 1$ to $n$
\par\qquad\qquad if $\exists k$ such that $S[k]=0$ and $k\le d_j$
\par\qquad\qquad then $i\leftarrow\max\{k|S[k]=0{\rm \ and\ }k\le d_j\}$
\par\qquad\qquad else $i\leftarrow\max\{k|S[k]=0\}$
\par\qquad\qquad\qquad $P\leftarrow P+p_j$
\par\qquad\qquad $S[i]\leftarrow j$

The worst-case time complexity of the greedy algorithm is $\Theta(n^2)$.

\item Proof of optimality

Assume that the optimal schedule looks different
from the greedy schedule. We wish to prove that the former can be
transformed to the latter without changing the total penalty. 
We use $c_j$ to denote the completion time of $J_j$ in the current schedule,
which is initially the optimal schedule.
Assume $p_1\ge p_2\ge\cdots\ge p_n$. For any $J_j$, it is either
an early job or a late job in the current schedule.
Consider the following procedure:

\qquad for $j\leftarrow 1$ to $n$
\par\qquad\qquad if $J_j$ is an early job in the current schedule
($c_j\le d_j$)
\par\qquad\qquad then $T\leftarrow\{J_i|J_i\in [c_j,d_j]{\rm\ and\ }
i>j\}$
\par\qquad\qquad\qquad if $T\not=\emptyset$
\par\qquad\qquad\qquad then let $J_i$ be the job in $T$ with the largest $c_i$
\par\qquad\qquad\qquad\qquad swap $J_i$ and $J_j$  //step 1
\par\qquad\qquad else //$J_j$ is a late job ($c_j>d_j$)
\par\qquad\qquad\qquad let $J_i$ be the latest scheduled job with 
$i>j$
\par\qquad\qquad\qquad swap $J_i$ and $J_j$  //step 2 

Step 1 does not increase the total penalty. This is because
after the swap $J_j$ remains to be an early job and $J_i$
is moved to an earlier slot. Step 2 also does not increase the
total penalty since $J_j$ remains to be a late job and $J_i$
is moved to an earlier slot. So after applying the above
procedure, the optimal schedule is transformed to a greedy
schedule while the total penalty is not increased. Since 
it is impossible to have a decreased total penalty, we have
a greedy schedule with the same total penalty as the optimal
schedule. Therefore, the greedy schedule is optimal.
 
\end{itemize}

\newpage

\section{Divide and Conquer}

\subsection{Introduction}

\noindent{\em Reading: CLRS 2.3}

\begin{itemize}

\item A general template:

\vskip 0.25cm
\begin{center}
\includegraphics[height=1.5in]{f8.pdf}
\end{center}

function $D\&C(x)$

\qquad if $x$ is small and or simple then return $adhoc(x)$

\qquad decompose $x$ into smaller instances $x_1,\ldots,x_k$ //$k\ge1$

\qquad for $i\leftarrow 1$ to $k$ $y_i\leftarrow D\&C(x_i)$

\qquad combine the $y_i$'s to obtain a solution $y$ for $x$

\qquad return $y$

{\em Remarks:} 

\begin{itemize}
\item Relations between $x$ and $x_1,\ldots,x_k$,
$y_1,\ldots,y_k$ and $y$; 

\item Time complexity $T(n)=\sum_{i=1}^k
T(n_i)+D(n)+C(n)$, where $D(n)$ and $C(n)$ are time for step 1 and
step 3, respectively; 

\item Trade-off between $D(n)$ and $C(n)$;

\item The time complexity requirement gives information on step 1
and step 3.
\end{itemize}

\begin{center}
\begin{tabular}{c|c}
$O(\log n)$ & $T(n)=T({n\over 2})+1$\\\hline
$O(n)$ & $T(n)=2T({n\over2})+1$ or $T({n\over2})+n$\\\hline
$O(n\log n)$ & $T(n)=2T({n\over2})+n$ or $T({n\over2})+n\log n$\\\hline
$O(n^2)$ & $T(n)=4T({n\over2})+n$ or $2T({n\over2})+n^2$
\end{tabular}
\end{center}
 
\item Examples of D\&C algorithms:

Binary search: $T(n)=T({n\over2})+O(1)$ $\Rightarrow$ $T(n)=O(\log n)$.

Merge sort: $T(n)=2T({n\over2})+O(n)$ $\Rightarrow$ $T(n)=O(n\log n)$,
where $D(n)=O(1)$ and $C(n)=O(n)$.

Quick sort (best case): $T(n)=2T({n\over2})+O(n)$ $\Rightarrow$ 
$T(n)=O(n\log n)$,
where $D(n)=O(n)$ and $C(n)=O(1)$.
 
Integer multiplication: $T(n)=3T({n\over2})+O(n)$ $\Rightarrow$
$T(n)=O(n^{\log 3})=O(n^{1.585})$, where $D(n)=O(n)$ and
$C(n)=O(n)$.

\end{itemize}

\subsection{Matrix multiplication}

\noindent{\em Reading: CLRS 28.2}

\begin{itemize}

\item Consider the multiplication of two $n\times n$ matrices.
Using the definition of matrix multiplication, we need $\Theta(n^3)$.
To use any divide-and-conquer idea, we have to first divide a matrix
into several smaller matrices. Let $A_{n\times n}$ and $B_{n\times n}$
be the two matrices. Let $C_{n\times n}=A_{n\times n}\cdot B_{n\times n}$.

$$A_{n\times n}=\left ( \begin{array}{cc}
A_{11} & A_{12}\\
A_{21} & A_{22}\\
\end{array} \right),\hskip 0.5in
B_{n\times n}=\left ( \begin{array}{cc}
B_{11} & B_{12}\\
B_{21} & B_{22}\\
\end{array} \right),\hskip 0.5in
C_{n\times n}=\left ( \begin{array}{cc}
C_{11} & C_{12}\\
C_{21} & C_{22}\\
\end{array} \right)$$

Then $C_{11}=A_{11}B_{11}+A_{12}B_{21}$, $C_{12}=A_{11}B_{12}+A_{12}B_{22}$,
$C_{21}=A_{21}B_{11}+A_{22}B_{21}$, and $C_{22}=A_{21}B_{12}+A_{22}B_{22}$.
So the multiplication of two $n\times n$ matrices becomes eight multiplications
of two ${n\over2}\times{n\over2}$ matrices, giving us $T(n)=8T({n\over2})+
\Theta(n^2)$. So $T(n)=\Theta(n^3)$ by master theorem. No improvement!

\item In the late sixties, Strassen reduced eight multiplications to seven.

\begin{center}
\begin{tabular}{l|l}
$M_1$ & $(A_{12}-A_{22})(B_{21}+B_{22})$\\\hline
$M_2$ & $(A_{11}+A_{22})(B_{11}+B_{22})$\\\hline
$M_3$ & $(A_{11}-A_{21})(B_{11}+B_{12})$\\\hline
$M_4$ & $(A_{11}+A_{12})B_{22}$\\\hline
$M_5$ & $A_{11}(B_{12}-B_{22})$\\\hline
$M_6$ & $A_{22}(B_{21}-B_{11})$\\\hline
$M_7$ & $(A_{21}+A_{22})B_{11}$
\end{tabular}
\hskip 0.5in
\begin{tabular}{l|l}
$C_{11}$ & $M_1+M_2-M_4+M_6$\\\hline
$C_{12}$ & $M_4+M_5$\\\hline
$C_{21}$ & $M_6+M_7$\\\hline
$C_{22}$ & $M_2-M_3+M_5-M_7$\\
\end{tabular}
\end{center}

Using the above idea in the divide-and-conquer algorithm,
we get $T(n)=7T({n\over2})+\Theta(n^2)$, thus $T(n)=\Theta(n^{\log 7})
=\Theta(n^{2.81})$ by master theorem.

\item In the late seventies, the matrix multiplication algorithm is
improved to $\Theta(n^{2.61})$ by Pan. In the late eighties,
the algorithm is improved to $\Theta(n^{2.376})$. There is still
a substantial gap to the $\Omega(n^2)$ lower bound.

\end{itemize}

\subsection{Finding the $k$th smallest}

\noindent{\em Reading: CLRS 8.2 and 8.3}

\begin{itemize}

\item Given a list of $n$ numbers, find the $k$th smallest number
among them.

\item First try: Sort the list in increasing order ($\Theta(n\log n)$)
and locate the $k$th element ($\Theta(1)$).

\item Second try: Similar to Quick Sort.
 
Function $Select(L,k)$

\qquad if $|L|<50$

\qquad then sort $L$ and return the $k$th

\qquad else choose any $p\in L$ as a pivot

\qquad\qquad $L_1=\{a_i\in L|a_i<p\}$

\qquad\qquad $L_2=\{a_i\in L|a_i=p\}$

\qquad\qquad $L_3=\{a_i\in L|a_i>p\}$

\qquad\qquad if $k\le |L_1|$ then return $Select(L_1,k)$

\qquad\qquad else if $k\le |L_1|+|L_2|$

\qquad\qquad\qquad then return $p$

\qquad\qquad\qquad else return $Select(L_3,k-|L_1|-|L_2|)$

Like Quick Sort, the time complexity of this algorithm heavily
depends on the selection of the pivot in each recursion.
If every time $p$ happens to partition $L$ evenly, then the
time complexity is $\Theta(n)$. However, if the partition is
extremely uneven, the time complexity degrades to $\Theta(n^2)$.
Therefore, the worst-case time of the algorithm is $\Theta(n^2)$.

\item Third try: Choose the pivot cleverly. Replace
``choose any $p\in L$ as a pivot'' in the above algorithm by the
following code:

\qquad\qquad divide $L$ into $\lfloor {|L|\over5}\rfloor$
sublists of (up to) $5$ elements each

\qquad\qquad sort each sublist into increasing order

\qquad\qquad let $M$ be the list of medians of all sublists

\qquad\qquad $p\leftarrow Select(M, \lceil {|M|\over2}\rceil)$

From the illustration below, there will be at most ${3\over4}|L|$
elements in $L_1$ and at most ${3\over4}|L|$ elements in $L_3$.
Therefore, $T(n)\le O(1)$ for $n\le49$ and 
$T(n)\le T({n\over5})+T({3n\over4})+O(n)$ for $n\ge 50$. By induction,
$T(n)=O(n)$. Since $\Omega(n)$ is obvious a lower bound, this
D\&C algorithm is optimal.

\vskip 0.25cm
\begin{center}
\includegraphics[height=1.5in]{f9.pdf}
\end{center}

\item Theorem: Let $T(n)\le cn$ for $n\le 49$ and $T(n)\le
T({n\over5})+T({3n\over4})+cn$ for $n\ge 50$. Show that
$T(n)\le 20cn$.

{\em Proof}\quad Induct on $n$. When $n\le 49$,
$T(n)\le cn\le 20cn$. Assume that $T(i)\le 20ci$ for $i\le n-1$.
Now consider $T(n)$.
\begin{eqnarray*}
T(n)&\le&T({n\over5})+T({3n\over4})+cn\\
&\le&20c{n\over5}+20c{3n\over4}+cn\\
&=&4cn+15cn+cn\\
&=&20cn
\end{eqnarray*}

\end{itemize}

\subsection{Exchanging two sections of an array}

\begin{itemize}

\item Given an array $A$ of $n$ items. How can one exchange the first
$k$ items with the last $n-k$ items?

\item A naive solution: copy the first $k$ elements to an auxiliary
array $B$; move the last $n-k$ elements to the first $n-k$ positions
of $A$; and copy the $k$ elements in $B$ back to the last $k$ positions
of $A$.

\item Assume only $\Theta(1)$ auxiliary memory is available. If the two sections have
the same length, it is easy. For example, $A=(a,b,c,d,e,f)$ and $k=3$.
We can exchange the sections by using just one additional variable:
swap $a,d$, swap $b,e$, and swap $c,f$.

\item A D\&C idea: Let $k=3$. $(a,b,c,d,e,f,g,h,i,j,k)\rightarrow
(d,e,f,g,h,i,j,k,a,b,c)$
 
\vskip 0.25cm
\begin{center}
\includegraphics[height=1.8in]{f10.pdf}
\end{center}

\item Algorithm: 

$Swap(i,j,m)$ // Assume $i+m\le j$

\qquad for $p\leftarrow 0 $ to $m-1$

\qquad\qquad swap $A[i+p]$ and $A[j+p]$

$Exchange(i,j,l,m)$ // Assume $i+l\le j$

\qquad if $l=m$ $Swap(i,j,l)$

\qquad else if $l<m$

\qquad\qquad\qquad $Swap(i,j+m-l,l)$

\qquad\qquad\qquad $Exchange(i,j,l,m-l)$

\qquad\qquad else

\qquad\qquad\qquad $Swap(i,j,m)$

\qquad\qquad\qquad $Exchange(i+m,j,l-m,m)$

\vskip 0.25cm
\begin{center}
\includegraphics[height=1.5in]{f11.pdf}
\end{center}

\item Time complexity: Let $T(l,m)$ be the number of single swaps.
If $l=m$, $T(l,m)=l$; if $l<m$, $T(l,m)=l+T(l,m-l)$; and
if $l>m$, $T(l,m)=m+T(l-m,m)$.

Prove by induction that $T(l,m)=l+m-gcd(l,m)$.

Induct on $l+m$ (assuming $l,m>0$). When $l+m=2$, $l=m=1$.
So $T(l,m)=T(1,1)=1$ by definition. On the other hand, 
$l+m-gcd(l,m)=2-gcd(1,1)=1$. So the claim holds. In the
inductive hypothesis, assume that for $l'+m'<l+m$,
$T(l',m')=l'+m'-gcd(l',m')$. Now consider the case of 
$l+m$. If $l=m$, $T(l,m)=l=2l-l=l+m-gcd(l,m)$. If $l<m$,
$T(l,m)=l+T(l,m-l)=l+(l+m-l)-gcd(l,m-l)=l+m-gcd(l,m-l)
=l+m-gcd(l,m)$. If $l>m$, $T(l,m)=m+T(l-m,m)=m+(l-m+m)-gcd(l-m,m)
=l+m-gcd(l,m)$.

\item $Exchange(1,k+1,k,n-k)$ solves the problem in time
$T(k,n-k)=n-gcd(k,n)$.

\end{itemize}

\subsection{Computing exponentiation}

\begin{itemize}

\item Idea:

$$a^n=\left\{\begin{array}{ll}
a & \mbox{if $n=1$}\\
(a^{n\over2})^2 & \mbox{if $n$ is even}\\
a\times a^{n-1} & \mbox{otherwise}
\end{array}\right.$$

For example, $a^{29}=aa^{28}=a(a^{14})^2=a((a^7)^2)^2=\cdots=
a((a(a(a)^2)^2)^2)^2$.
It takes a total of seven multiplications.

For simplicity, we call this algorithm $expodac(a,n)$.

\item Time complexity analysis:

Let $N(n)$ be the number of multiplications in $expodac(a,n)$.

$$N(n)=\left\{\begin{array}{ll}
0 & \mbox{if $n=1$}\\
N({n\over2})+1 & \mbox{if $n$ is even}\\
N(n-1)+1 & \mbox{otherwise}
\end{array}\right.$$

When $n>1$ is odd, $N(n)=N(n-1)+1=N({n-1\over2})+2=N(\lfloor{n\over2}
\rfloor)+2$. When $n$ is even, $N(n)=N({n\over2})+1=N(\lfloor{n\over2}
\rfloor)+1$. Therefore, $N(\lfloor{n\over2}\rfloor)+1\le N(n)\le
N(\lfloor{n\over2}\rfloor)+2$. Define two functions $N_i$ for
$i=1,2$ as follows.

$$N_i(n)=\left\{\begin{array}{ll}
0 & \mbox{if $n=1$}\\
N_i(\lfloor{n\over2}\rfloor)+i & \mbox{otherwise}
\end{array}\right.$$

It is easy to prove that $N_1(n)\le N(n)\le N_2(n)$. Since
$N_1(n)=\Theta(\log n)$ and $N_2(n)=\Theta(\log n)$, then
$N(n)=\Theta(\log n)$.

Now, if each integer multiplication can be done in constant time, 
the time complexity of $expodac(a,n)$ is $\Theta(\log n)$. But
what if the integers involved are so large that a multiplication
can not be completed in constant time?
Let $M(p,q)$ be the time needed to multiply two integers of
sizes $p$ and $q$ (the numbers of figures). 
Let $T(p,n)$ be the time complexity of $expodac(a,n)$,
where $p$ is the size of $a$.

Theorem/Exercise: The size of $a^i$ is at least $i(p-1)$ but at most $ip$.

Inspection of $expodac(a,n)$ yields the following definition of $T(p,n)$.

$$T(p,n)\le\left\{\begin{array}{ll}
0 & \mbox{if $n=1$}\\
T(p,{n\over2})+M({pn\over2},{pn\over2}) & \mbox{if $n$ is even}\\
T(p,n-1)+M(p,p(n-1)) & \mbox{otherwise}
\end{array}\right.$$

If $n>1$ is odd, $T(p,n)\le T(p,n-1)+M(p,p(n-1))\le
T(p,{n-1\over2})+M({p(n-1)\over2},{p(n-1)\over2})+M(p,p(n-1))
=T(p,\lfloor{n\over2}\rfloor)+M(p\lfloor{n\over2}\rfloor,
p\lfloor{n\over2}\rfloor)+M(p,p(n-1))$. If $n$ is even,
$T(p,n)=T(p,{n\over2})+M({pn\over2},{pn\over2})
\le T(p,\lfloor{n\over2}\rfloor)+M(p\lfloor{n\over2}\rfloor,
p\lfloor{n\over2}\rfloor)+M(p,p(n-1))$. So in both cases,

$$T(p,n)\le T(p,\lfloor{n\over2}\rfloor)+M(p\lfloor{n\over2}\rfloor,
p\lfloor{n\over2}\rfloor)+M(p,p(n-1)).$$

In general, $M(p,q)=\Theta(qp^{\alpha-1})$ where $p\le q$ and
$\alpha=2$ in the classic integer multiplication algorithm and $\alpha=\log 3$ in the
divide-and-conquer algorithm. So $M(p\lfloor{n\over2}\rfloor,
p\lfloor{n\over2}\rfloor)=\Theta((p\lfloor{n\over2}\rfloor)^{\alpha})$
and $M(p,p(n-1))=\Theta(p^{\alpha}(n-1))$. So,

$$T(p,n)\le T(p,\lfloor{n\over2}\rfloor)+\Theta(p^{\alpha}n^{\alpha}).$$

By the iteration method or the master method, we have
$T(p,n)\le \Theta(p^{\alpha}n^{\alpha})$, thus
$T(p,n)=O(p^{\alpha}n^{\alpha})$.

On the other hand, consider the last or the next to last multiplication
in $expodac(a,n)$, depending on whether $n$ is even or odd. It involves
squaring $a^{\lfloor{n\over2}\rfloor}$, which is of size at least
$(p-1)\lfloor{n\over2}\rfloor$. So,

$$T(p,n)\ge M((p-1)\lfloor{n\over2}\rfloor,(p-1)\lfloor{n\over2}\rfloor).$$

Since $M((p-1)\lfloor{n\over2}\rfloor,(p-1)\lfloor{n\over2}\rfloor)
=\Theta(((p-1)\lfloor{n\over2}\rfloor)^{\alpha})$, so
$T(p,n)\ge\Theta(p^{\alpha}n^{\alpha})$, thus
$T(p,n)=\Omega(p^{\alpha}n^{\alpha})$.

Combining the two bounds, we have $T(p,n)=\Theta(p^{\alpha}n^{\alpha})$,
where $\alpha=2$ if the classic algorithm is used and $\alpha=\log 3$
if the divide-and-conquer algorithm is used.

\end{itemize}

\subsection{The closest-pair problem}

\noindent{\em Reading: CLRS 33.4 }

\begin{itemize}

\item Problem:

Given $n$ 2D points, find the two closest points.

{\em Remarks:}
\begin{itemize}
\item Input is $(x_1,y_1),\ldots,(x_n,y_n)$ and output is
$(x_i,y_i),(x_j,y_j)$.
\item The distance between $p_1=(x_1,y_1)$ and $p_2=(x_2,y_2)$ is
$|p_1p_2|=\sqrt{(x_1-x_2)^2+(y_1-y_2)^2}$.
\item A brute-force algorithm: ${n\choose2}=O(n^2)$.
\end{itemize}

\item  A divide-and-conquer algorithm:

Let $X$ be point set $P$ sorted by increasing $x$ and
$Y$ be point set $P$ sorted by increasing $y$.

\begin{enumerate}

\item Divide: Use a vertical line $l$ to bisect $P$ into $P_L$ and $P_R$.
$X$ and $Y$ are thus correspondingly partitioned into $X_L$ and $X_R$,
$Y_L$ and $Y_R$.

\item Conquer: Use two recursive calls to find the closest pairs in
$P_L$ and $P_R$. Let $\delta_L$ ($\delta_R$) be the distance between the
two closest points in $P_L$ ($P_R$). Define $\delta=\min\{\delta_L,
\delta_R\}$.

\item Merge: Find the closest pair in $P$. It can be the pair
with distance $\delta$ found in the previous step, or a pair
with one point in $P_L$ and the other in $P_R$ with a distance
less than $\delta$.

\end{enumerate}

\item Time complexity: $T(n)=T_P(n)+T_{DAC}(n)$.

\begin{itemize}
\item Preprocessing:
Sort $P$ twice to construct $X$ and $Y$. $\Rightarrow$ $O(n\log n)$
\item Divide: Use the median in $X$ to create the partition of $P$
into $P_L$ and $P_R$. Construct the partition of $X$ into $X_L$
and $X_R$. (This is easy.) Construct the partition of $Y$ into
$Y_L$ and $Y_R$. (This can be tricky.) All of the above must be
done in linear time. To check whether you have the right partitions,
are $P_L$, $X_L$, and $Y_L$ the same point set, and are $P_R$,
$X_R$, and $Y_R$ the same point set? $\Rightarrow$ $O(n)$
\item Conquer:
Two recursive calls on point sets of size $n\over2$. $\Rightarrow$
$2T_{DAC}({n\over2})$
\item Merge: Many technical details to fill in. We wish to spend
only linear time
for the merge. Can we achieve this goal? $\Rightarrow$ $O(n)$
\end{itemize}

So $T_{DAC}(n)=2T_{DAC}({n\over2})+O(n)=O(n\log n)$.
Overall, $T(n)=T_P(n)+T_{DAC}(n)=O(n\log n)+O(n\log n)=O(n\log n)$.

\item Merge in linear time:

Assume that
\begin{itemize}
\item $P\Rightarrow P_L,P_R$, $X\Rightarrow X_L,X_R$,
$Y\Rightarrow Y_L,Y_R$ by the vertical line $l$.
\item $\delta_L$ is the distance between the closest points in
$P_L$ and $\delta_R$ is the distance between the closest points
in $P_R$.
\item $\delta=\min\{\delta_L,\delta_R\}$.
\end{itemize}

Goal: Determine if there are two points, one in $P_L$ and the other
in $P_R$, with distance less than $\delta$.

An exhaustive search checks all pairs and may take $O(n^2)$.
Can we just check $O(n)$ pairs and not miss any one with distance
less than $\delta$? Yes and here is why.

Define a strip centered at $l$ with width $2\delta$. Let $P_S$ be
the set of points in the strip and $Y_S$ be $P_S$ sorted by $y$.
(Remember our linear time restriction: Can you create $P_S$ and $Y_S$
in linear time?)

Claim: If there are points $p\in P_L$ and $q\in P_R$ with $|pq|<\delta$,
then $p$ and $q$ must be in the strip.

Pause: Can we check out all pairs in $P_S$ to determine the one with
the smallest distance?

For each $p\in Y_S$, define a rectangle $R(p)$ of height $\delta$
and width $2\delta$, with the bottom edge of the rectangle passing $p$.

Claim: If there are $p$ and $q$ with $|pq|<\delta$ and $q.y\ge p.y$,
$q$ must be in $R(p)$.

Claim: There can be at most eight points in each $R(p)$.

Why? Divide $R(p)$ ($\delta\times2\delta$) into eight
${\delta\over2}\times{\delta\over2}$ squares. In each square,
if there are two or more points, say $q_1$ and $q_2$, then
$$|q_1q_2|\le{\rm diagonal\ of\ the\ square}={\sqrt 2}{\delta\over2}
<\delta,$$
which is impossible since $q_1$ and $q_2$ are on the same side of the
vertical line $l$. So there can be at most one point in each square,
with a total of eight points in $R(p)$.

Claim: For any $p\in Y_S$, if there is $q$ such that $|pq|<\delta$,
then $q$ must be one of the seven points following $p$ in $Y_S$.

Algorithm for merge:

\begin{verbatim}
m = |Ys|
mindist = |Ys[0]Ys[1]|
p = Ys[0]
q = Ys[1]
for i = 1 to m-1
  k = min {i + 7, m}
  for j = i + 1 to k
    dist = |Ys[i]Ys[j]|
    if dist < mindist
      mindist = dist
      p = Ys[i]
      q = Ys[j]
If mindist < delta
  return p and q as the closest points
else return the closest points found by
     the recursive calls
\end{verbatim}

\end{itemize}

\newpage

\section{Dynamic Programming}

\subsection{Introduction}

\noindent{\em Reading: CLRS 15.3}

\begin{itemize}

\item Divide-and-conquer algorithms are implemented by recursion. Its
design is top-down, and it is efficient when the subproblems
don't overlap. However, when subproblems do overlap (share
sub-subproblems), recursion does redundant work. In this case,
a tabular method is often used. It is nonrecursive and bottom-up. 
It is called dynamic programming.

\item An example: Fibonacci numbers:

$fib1(n)$

\qquad if $n<2$ return $n$

\qquad else return $fib1(n-1)+fib1(n-2)$

We can see that this recursive (divide-and-conquer) algorithm is
not efficient. To compute $fib1(n)$, the algorithm computes
$fib1(n-1)$ and $fib1(n-2)$ separately. To compute $fib1(n-1)$,
the values of $fib1(n-2)$ and $fib1(n-3)$ are needed. To compute
$fib1(n-2)$, the values of $fib1(n-3)$ and $fib1(n-4)$ are needed.
We observe that subproblems $fib1(n-1)$ and $fib(n-2)$ share
sub-subproblem. 

The time complexity $T(n)\ge T(n-1)+T(n-2)$. So $T(n)$ is larger than
the $n$th Fibonacci number. So $T(n)\ge {1\over\sqrt5}
(({1+\sqrt5\over2})^n-(-{1+\sqrt5\over2})^{-n})=\Theta({1.618}^n)$.

We can use the dynamic programming method by building a
1-D table as below and returning the $n$th entry of the table.

\begin{center}
\begin{tabular}{c|ccccccc}
$k$ & $0$ & $1$ & $2$ & $3$ & $4$ & $\cdots$ & $n$\\\hline
$f_k$ & $0$ & $1$ & $1$ & $2$ & $3$ & $\cdots$ & $f_n$\\
\end{tabular}
\end{center}

$fib2(n)$

\qquad if $n<2$ return $n$

\qquad else $i\leftarrow 0$

\qquad\qquad $j\leftarrow 1$

\qquad\qquad for $k\leftarrow 2$ to $n$ 

\qquad\qquad\qquad $f\leftarrow i+j$

\qquad\qquad\qquad $i\leftarrow j$

\qquad\qquad\qquad $j\leftarrow f$

\qquad\qquad return $f$

The time complexity is obviously $O(n)$.

To summarize, to use dynamic programming, first define
a function $F$ recursively (so that the solution information
is embedded in $F(n)$): $F(n)=G(F(n_1), F(n_2), \ldots, F(n_k))$
for $n_1,n_2,\ldots,n_k<n$. Construct a table to compute nonrecursively
$F(n_1), F(n_2),\ldots,F(n_k)$, hence $F(n)$.

\end{itemize}

\subsection{Traveling salesman problem (TSP)}

\begin{itemize}

\item Given a edge-weighted graph $G=(V,E)$. Find a tour (a cycle that
passes through each vertex exactly once) with the minimum total
weight. 

\item Assume $V=\{1,2,\ldots,n\}$. Let $S\subseteq
\{2,3,\ldots,n\}$ and $i\not\in S$.
Define $C(S,i)=$ minimum total weight of simple paths from $1$ to
all nodes in $S$ and to $i$. First, $C(\emptyset,i)=w(1,i)$,
where $w(1,i)=\infty$ if $(1,i)\not\in E$.
Second, $C(S,i)=\min_{k\in S}\{C(S-\{k\},k)+w(k,i)\}$ if
$S\not=\emptyset$. Then the total weight of the traveling
salesman tour is $C(\{2,3,\ldots,n\},1)$.

\item If using recursion to compute $C$, 
\begin{eqnarray*}
T(n)&\ge&(n-1)(T(n-1)+1)\\
&=&(n-1)T(n-1)+(n-1)=(n-1)(n-2)T(n-2)+(n-1)(n-2)+(n-1)\\
&=&\cdots\\
&=&(n-1)!T(1)+(n-1)!+{(n-1)!\over1!}+{(n-1)!\over2!}+\cdots+
{(n-1)!\over (n-2)!}\\
&=&(n-1)!(1+1+{1\over1!}+{1\over2!}+{1\over3!}+\cdots+{1\over(n-2)!})\\
&>&3(n-1)!\\
&=&\Omega((n-1)!).
\end{eqnarray*}

\item If using dynamic programming, 

\begin{center}
\begin{tabular}{r|cccc}
$S\backslash i$ & $1$ & $2$ & $\cdots$ & $n$\\\hline
$\emptyset$ & $\rightarrow$ & $\rightarrow$ & $\cdots$ & $\rightarrow$\\
$\cdots$ & $\cdots$ & $\cdots$ & $\cdots$ & $\cdots$ \\
$\{2,3,\ldots,n\}$ & $*$ & $\cdots$ & $\cdots$ & $\cdots$\\
\end{tabular}
\end{center}

\item The arrows indicate the computing order: row by row and left to right.
The number of rows is the same as the number of subsets of $\{2,3,\ldots,n\}$,
which is ${n-1\choose 0}+{n-1 \choose 1}+\cdots+{n-1 \choose{n-1}}=2^{n-1}$.
Also there are $n$ columns. To compute each entry $C[S,i]$, we need
$O(|S|)=O(n)$ steps. So the total time needed is $T(n)=O(n^22^{n-1})$. An
improvement over the recursive implementation.

\end{itemize}

\subsection{Chained matrix multiplication}

\noindent{\em Reading: CLRS 15.2}

\begin{itemize}

\item We wish to compute $A_1\times A_2\times\cdots\times A_n$, where
$A_i$ is a $p_{i-1}\times p_i$ matrix. Which order of computation should
we use to achieve the highest efficiency of the algorithm?

\item The number of basic operations needed to compute $A_i\times A_{i+1}$ is 
$p_{i-1}p_ip_{i+1}$. 

\item Order of computation determines the time efficiency. For example,
$A_1: 10\times 20$, $A_2: 20\times 50$, $A_3: 50\times 1$, and
$A_4: 1\times 100$. If we use the order in 
$A_1\times(A_2\times(A_3\times A_4)$, the number of basic operations
is $(50\times1\times100)+(20\times50\times100)+(10\times20\times100)
=125,000$. However, if we use the order in 
$((A_1\times A_2)\times A_3)\times A_4$, the number of basic operations
is $(10\times20\times50)+(10\times50\times1)+(10\times1\times100)
=11,500$.

\item Question: What is the minimum number of basic operations
in computing $A_1\times A_2\times\cdots\times A_n$?

\item Let $m(i,j)$ be the minimum number of basic operations in
computing $A_i\times A_{i+1}\times\cdots\times A_j$ for
$1\le i\le j\le n$. Assume in general that $k$ is used to
indicate the position of the last multiplication to be 
performed among all: $(A_i\times\cdots\times A_k)\times
(A_{k+1}\times\cdots\times A_j)$. Then

$m(i,j)=0$ if $i=j$.

$m(i,j)=\min_{i\le k\le j-1}\{m(i,k)+m(k+1,j)+p_{i-1}p_kp_j\}$ if $i\not=j$.

\item We can use a dynamic programming algorithm to compute $m(1,n)$,
the minimum number of basic operations in computing 
$A_1\times A_2\times\cdots\times A_n$. Entries are filled left to right
and bottom to top. Note that those in the lower left triangle are
undefined.

\begin{center}
\begin{tabular}{c|c|c|c|c}
$i\backslash j$ & $1$ & $2$ & $\cdots$ & $n$\\\hline
$1$ & $0$ & $\uparrow$ & $\cdots$ & $*$\\
$2$ & $--$ & $0$ & $\cdots$ & $\uparrow$\\
$\cdots$ & $\cdots$ & $\cdots$ & $\cdots$ & $\cdots$\\
$n$ & $--$ & $--$ & $\cdots$ & $0$\\
\end{tabular}
\end{center}

\item The algorithm:

\qquad for $i\leftarrow 1$ to $n$ 

\qquad\qquad $m[i,i]\leftarrow 0$

\qquad for $j\leftarrow 2$ to $n$

\qquad\qquad for $i\leftarrow j-1$ to $1$

\qquad\qquad\qquad $m[i,j]\leftarrow\min_{i\le k\le j-1}
\{m[i,k]+m[k+1,j]+p_{i-1}p_kp_j\}$

\item Time complexity: $O(n^3)$.

\end{itemize}

\subsection{Making change}

\begin{itemize}

\item Let $n$, a positive integer, be the number of different
types of coin in a country. Let $coin[1..n]$, an array of
positive integers, be the values of these $n$ types of coin. Let
$m$, a positive integer, be the amount of change that one wishes to make.
Design a dynamic programming algorithm that determines whether $m$
can be made with the coins, and if so, computes the minimum number of coins
needed.

\item Define $count(i)$ to be the minimum number of coins to
make $i$ ($>0$). That $count(i)=\infty$ implies that no solution
exists. The recursive definition of $count(i)$ is as follows.

$count(1)=\infty$ if $1\not\in coin[\ ]$.

$count(coin[j])=1$ for $j=1,\ldots,n$.

$count(i)=1+\min_{1\le j\le n,coin[j]<i}\{count(i-coin[j])\}$

\item The table is a 1-D table and its entries are filled from left
to right until $count[m]$ is reached.

\begin{center}
\begin{tabular}{c|c|c|c|c}
$i$ & $1$ & $2$ & $\cdots$ & $m$\\\hline
$count[i]$ & $\rightarrow$ & $\rightarrow$ & $\cdots$ & $*$\\
\end{tabular}
\end{center}

\item Algorithm:

\qquad for $i\leftarrow 1$ to $m$ $count[i]\leftarrow -1$

\qquad $count[1]\leftarrow\infty$

\qquad for $j\leftarrow 1$ to $n$

\qquad\qquad $count[coin[j]]\leftarrow 1$

\qquad for $i\leftarrow 1$ to $m$

\qquad\qquad if $count[i]=-1$

\qquad\qquad\qquad $min\leftarrow\infty$

\qquad\qquad\qquad for $j\leftarrow 1$ to $n$

\qquad\qquad\qquad\qquad if $coin[j]<i$ 
              
\qquad\qquad\qquad\qquad\qquad if $min>count[i-coin[j]]$

\qquad\qquad\qquad\qquad\qquad\qquad $min\leftarrow count[i-coin[j]]$

\qquad\qquad\qquad $count[i]\leftarrow 1+min$
 
\item Time complexity: $\Theta(mn)$. (pseudo-polynomial)

\end{itemize}

\subsection{Longest common subsequence}

\noindent{\em Reading: CLRS 15.4}

\begin{itemize}

\item Subsequence: If $X=<A,B,C,B,D,A,B>$ and $Z=<B,C,D,B>$, then
$Z$ is a subsequence of $X$.

Common subsequence: Let $Y=<B,D,C,A,B,A>$. Then $<B,C,A>$ is a
common subsequence of $X$ and $Y$. 

Longest common subsequence (LCS): For $X$ and $Y$, there is no 
common subsequence with length longer than $4$. $<B,C,B,A>$ and
$<B,D,A,B>$ are both LCS's of $X$ and $Y$.

Question: Given two sequences, what is the length of their LCS?
(What is the LCS of the sequences?)

\item A brute-force method:

Assume $X=<x_1,\ldots,x_m>$ and $Y=<y_1,\ldots,y_n>$. For each subsequence
of $X$, check if it is also a subsequence of $Y$, keeping track of the
longest found.

How many possible subsequences are there for $X$?
${m \choose 0}+{m \choose 1}+{m \choose 2}+\cdots+{m \choose m}
=2^m$.

\item A recursive approach:

Define $X_i=<x_1,\ldots,x_i>$ and $Y_j=<y_1,\ldots,y_j>$.
Define $C(i,j)$ to be the length of the LCS of $X_i$ and $Y_j$.

$C(i,j)=0$ if $i=0$ or $j=0$;

$C(i,j)=C(i-1,j-1)+1$ if $i,j>0$ and $x_i=y_j$;

$C(i,j)=\max\{C(i,j-1),C(i-1,j)\}$ if $i,j>0$ and $x_i\not=y_j$.

When $x_i=y_j$, $X_i=X_{i-1}<x_i>$ and $Y_j=Y_{j-1}<y_j>$.
So $LCS(X_i,Y_j)=LCS(X_{i-1},Y_{j-1})x_i$. Hence,
$C(i,j)=C(i-1,j-1)+1$.

When $x_i\not=y_j$, $x_i$ and $y_j$ cannot both appear in
$LCS(X_i,Y_j)$. So $LCS(X_i,Y_j)=LCS(X_i,Y_{j-1})$ or
$LCS(X_{i-1},Y_j)$. Hence, $C[i,j]=\max\{C(i,j-1),C(i-1,j)\}$.

\item A nonrecursive implementation: Dynamic programming:

A 2-D table is constructed where each entry is filled left to
right and top to bottom. The initialization handles the first row
and the first column of the table. Entry $C[m,n]$ is the length
of the LCS of $X$ and $Y$.

\begin{center}
\begin{tabular}{c|ccccc}
$i\backslash j$ & $0$ & $1$ & $2$ & $\cdots$ & $n$\\\hline
$0$ & $0$ & $0$ & $0$ & $\cdots$ & $0$\\
$1$ & $0$ & $\rightarrow$ & $\rightarrow$ & $\cdots$ & $\rightarrow$\\
$2$ & $0$ & $\rightarrow$ & $\rightarrow$ & $\cdots$ & $\rightarrow$\\
$\cdots$ & $\cdots$ & $\cdots$ & $\cdots$ & $\cdots$ & $\cdots$\\
$m$ & $0$ & $\rightarrow$ & $\rightarrow$ & $\cdots$ & $*$\\
\end{tabular}
\end{center}

\item The algorithm:

\qquad for $i\leftarrow 0$ to $m$ $C[i,0]\leftarrow 0$

\qquad for $j\leftarrow 0$ to $n$ $C[0,j]\leftarrow 0$

\qquad for $i\leftarrow 1$ to $m$

\qquad\qquad for $j\leftarrow 1$ to $n$

\qquad\qquad\qquad if $x_i=y_j$ $C[i,j]\leftarrow C[i-1,j-1]+1$

\qquad\qquad\qquad else $C[i,j]=\max\{C[i,j-1],C[i-1,j]\}$

\item Time complexity: $\Theta(mn)$.

\item How to compute the LCS in addition to the length of the LCS:
Maintain an array $S[i,j]$ of special characters. Set $S[i,0]=S[0,j]=
\sqcup$ (single space) for $0\le i\le m$ and $0\le j\le n$.
In the nested for loop, if $x_i=y_j$, set $S[i,j]$ to be
$\nwarrow$, else if $C[i,j-1]\ge C[i-1,j]$, set $S[i,j]$ to be
$\leftarrow$, and if $C[i,j-1]<C[i-1,j]$, set $S[i,j]$ to be
$\uparrow$. The following additional code generates the
LCS of two sequences.

\qquad $i\leftarrow m$

\qquad $j\leftarrow n$

\qquad while $S[i,j]\not=\sqcup$

\qquad\qquad if $S[i,j]=\leftarrow$ $j\leftarrow j-1$

\qquad\qquad else if $S[i,j]=\uparrow$ $i\leftarrow i-1$

\qquad\qquad else push $x_i$ to a stack

\qquad\qquad\qquad $i\leftarrow i-1$

\qquad\qquad\qquad $j\leftarrow j-1$

\qquad output the content in the stack

\end{itemize}

\subsection{Optimal binary search tree}

\noindent{\em Reading: CLRS 15.5}

\begin{itemize}

\item Given a set of keys (numbers) and the probability that each key
is located. How can one organize the set in a binary search tree so that
the average time to locate a key in the tree is minimized?

\item For each node (key) in a binary search tree, the time needed to
locate the node is its level number. 

\item Let the keys be $a_1, a_2,\ldots,a_n$ (in increasing order).
Let $l_i$ be the level number of the node corresponding to key $a_i$
in a given binary search tree. Let $p_i$ be the probability that
$a_i$ is to be located. Then the average search time for that tree
is $\sum_{i=1}^n p_il_i$. We wish to build an optimal binary search
tree, where this cost is minimized.

\item An example: $n=3$ and $p_1=0.7$, $p_2=0.2$ and $p_3=0.1$.
The following figure contains all five possible binary search trees
for $n=3$.

\vskip 0.25cm
\begin{center}
\includegraphics[height=1in]{f12.pdf}
\end{center}

1. 3(0.7)+2(0.2)+1(0.1)=2.6

2. 2(0.7)+3(0.2)+1(0.1)=2.1

3. 2(0.7)+1(0.2)+2(0.1)=1.8

4. 1(0.7)+3(0.2)+2(0.1)=1.5

5. 1(0.7)+2(0.2)+3(0.1)=1.4 $\Leftarrow$ optimal!

\item A recursive approach: Let $c(i,j)$ be the average search time
in a tree with only $a_i,\ldots,a_j$, where $1\le i\le j\le n$.
If $a_k$ happens to be the root of the tree containing $a_i,\ldots,a_j$,
then in the left subtree are $a_i,\ldots,a_{k-1}$ and in
the right subtree are $a_{k+1},\ldots,a_j$.

$c(i,i)=p_i$ for $1\le i\le n$

$c(i,j)=\min_{i\le k\le j}\{c(i,k-1)+c(k+1,j)+\sum_{l=i}^jp_l\}$ for $i<j$

$c(i,j)=0$ for $i=j+1$ (Why needed?)

\item A dynamic programming algorithm with $O(n^3)$:

\begin{center}
\begin{tabular}{c|c|c|c|c|c}
$i\backslash j$ & $1$ & $2$ & $\cdots$ & $n-1$ & $n$\\\hline
$1$ & $p_1$ & $\uparrow$ & $\cdots$ & $\uparrow$ & $*$\\
$2$ & $0$ & $p_2$ & $\cdots$ & $\uparrow$ & $\uparrow$\\
$\cdots$ & $\cdots$ & $\cdots$ & $\cdots$ & $\cdots$ & $\cdots$\\
$n$ & $--$ & $--$ & $\cdots$ & $0$ & $p_n$\\
\end{tabular}
\end{center}
 
\end{itemize}

\subsection{Memory functions}

\noindent{\em Reading: CLRS 15.3}

\begin{itemize}

\item Divide and conquer: Only needed entries are 
computed but some entries are computed more than once.
Dynamic programming: All entries
in the table are computed once, whether needed or not.

\item A compromise: Only compute needed entries exactly once.
To do so, we combine the recursive implementation with a
table. Before we enter a recursion, we check the table to
see whether the entry has been computed before. This method
is called the memory function method.

\item Example: Chained matrix multiplication revisited.

We first initialize all entries in table $m[1..n,1..n]$ to be $-1$,
and then call $mf(1,n)$.

$mf(i,j)$

\qquad if $i=j$ return 0

\qquad if $m[i,j]\not=-1$ return $m[i,j]$

\qquad $c\leftarrow\infty$

\qquad for $k\leftarrow i$ to $j-1$

\qquad\qquad $c\leftarrow\min\{c, mf(i,k)+mf(k+1,j)+p_{i-1}p_kp_j\}$

\qquad $m[i,j]\leftarrow c$

\qquad return $c$

\item The time complexity is no larger than that in the
corresponding dynamic programming algorithm, but the space
complexity will be more since recursion requires more space 
to implement.

\end{itemize}

\newpage

\section{The Lower Bound Theory}

\subsection{What is a lower bound?}

\begin{itemize}

\item Algorithmics/upper bounds:

Prove that a problem can be solved in time $O(f(n))$ by designing
and analyzing a specific algorithm for the problem, for some
$f(n)$ that we aim to reduce as much as possible. We say that 
$O(f(n))$ is an upper bound to the problem. An upper bound is
always paired with an algorithm. It is the amount of time sufficient
to solve the problem.

\item Complexity/lower bounds:

Prove that any algorithm capable of solving the problem correctly
on all of its instances must take time $\Omega(g(n))$, for some 
$g(n)$ that we try to push as large as possible. We say that 
$g(n)$ is a lower bound to the problem. A lower bound shows the
difficult nature of the problem, so it has nothing to do with any
specific algorithm. It is the amount of time necessary to solve
the problem by any algorithm within the model.

\item Optimality of an algorithm:

If $f(n)=\Theta(g(n))$, then we say that we have found the most efficient 
algorithm possible, except perhaps for changes in the hidden multiplicative
constant.

\item A graphical explanation:

\vskip0.25cm
\begin{center}
\includegraphics[height=1.2in]{f13.pdf}
\end{center}

\item Example: Matrix multiplication.

Lower bound: $n^2$

Upper bounds: $n^3\rightarrow n^{2.81}\rightarrow n^{2.61}\rightarrow
n^{2.376}\rightarrow\cdots$
   
\end{itemize}

\subsection{The decision tree method}

\noindent{\em Reading: CLRS 8.1}

\begin{itemize}

\item A comparison-based model for proving worst-case lower bound:

The following is a decision tree for sorting 3 items by comparisons.
Each interior node represents a comparison. Each tree edge is an
outcome ($<$ or $>$). Each leaf is a possible output (a sorted list).
Each path from the root to a leaf represents the steps of sorting
a certain type of input. There are $n!$ leaves in the tree,
corresponding to $n!$ permutations. The height of the tree gives
the maximum number of comparisons in the algorithm for all inputs,
thus the worst case time complexity of the algorithm the tree 
represents.

How to use a decision tree to establish a lower bound: For each
comparison-based sorting algorithm, $A_i$, there is a decision
tree, $T_i$. Let $height(T_i)$ be the height of $T_i$, thus the
worst-case time of $A_i$. Obviously, $\min_i\{height(T_i)\}$ is
a lower bound to the problem.

\vskip 0.25cm
\begin{center}
\includegraphics[height=1.3in]{f14.pdf}
\end{center}

\item Lemma: A binary tree of height $h$ has at most $2^h$ leaves.

{\em Proof}\quad We assume that the height of a tree is the number of edges
on the longest path between the root and a leaf. We induct on $h$.
When $h=0$, the tree has just one node. There is 1 leaf, which is no larger
than $2^0$. Assume for $i\le h-1$, a tree of height $i$ has no more
than $2^i$ leaves. Now consider a tree with height $h$. Assume the
heights of the left and right subtrees are $h_L$ and $h_R$, respectively.
so $h=\max\{h_L,h_R\}+1$. By the inductive hypothesis, the
numbers of leaves in the left and right subtrees are at most
$2^{h_L}$ and $2^{h_R}$, respectively. So the number of leaves in the
tree with height $h$ is at most $2^{h_L}+2^{h_R}\le 2^h$.

\item Theorem: Any comparison-based sorting algorithm requires
$\Omega(n\log n)$ time in the worst case.

{\em Proof}\quad Let $T$ be a decision tree for any algorithm that
sorts $n$ elements by comparisons. $T$ has $n!$ leaves. So the height
of the tree, $h$, is at least $\log n!$. (Assume not. Then $h<log n!$.
By the lemma, the number of leaves in the tree is at most $2^h<2^{log n!}
=n!$, which is impossible.) So for any sorting algorithm, the worst-case
time complexity is the height of the decision tree, thus at least $\log n!
=\Omega(n\log n)$.
 
\item Both Merge Sort and Heap Sort achieve the optimality of sorting
since their worst-case times match the established lower bound, $n\log n$.
Radix Sort has a worst-case time of $\Theta(d(n+k))$, where $d$ is the number of
digits and $k$ is the base. It is $\Theta(n)$ when $d=O(1)$ and $k=O(n)$.
This is not a contradiction to the $n\log n$ lower bound, since Radix
Sort is not comparison based. 

\item A comparison-based model for proving average-case lower bound:

Again consider the example of sorting 3 elements. As in
any average-case analysis, assume that all elements in the list
are distinct and that all permutations are equally likely.
For any sorting algorithm, there is a corresponding decision tree
with $n!$ leaves. Let $d_i$ be the depth, or the number of edges on the
path between the root and a leaf $i$. Let $p_i$ be the probability 
that permutation (leaf) $i$ occurs. Then the average-case time
of the algorithm that the tree represents is $\sum_i p_id_i={1\over n!}
\sum_id_i$.

\begin{center}
\begin{tabular}{c|c|c|c|c|c|c}
leaf & $abc$ & $acb$ & $cab$ & $bac$ & $bca$ & $cba$\\\hline
probability & $p_1$ & $p_2$ & $p_3$ & $p_4$ & $p_5$ & $p_6$\\
depth & $d_1$ & $d_2$ & $d_3$ & $d_4$ & $d_5$ & $d_6$\\
\end{tabular}
\end{center}

\item Lemma: Let $T_m$ be any binary tree with $m$ leaves such that each nonleaf
has exactly 2 children. Define $D(T_m)=\sum_id_i$ (the sum of the depths of all
leaves in $T$). Define $D_m=\min_{\forall T_m}\{D(T_m)\}$. Prove that
$D_m\ge m\log m$.

{\em Proof}\quad We induct on $m$. When $m=1$, the tree has only one node.
So $D_1=0\ge 1\log 1$. Assume that for any $i\le m-1$, $D_i\ge i\log i$.
Now consider any tree with $m$ leaves. Assume there are $i$ leaves in the
left subtree and $m-i$ leaves in the right subtree. ($i\not=0$ and
$i\not=m$ since each nonleaf has two children.) So 
\begin{eqnarray*}
D_m&=&\min_{1\le i\le m-1}\{D_i+D_{m-i}+m\}\\
&\ge& m+\min_{1\le i\le m-1}\{i\log i+(m-i)\log (m-i)\}
\end{eqnarray*}
Define function $f(x)=x\log x+(m-x)\log (m-x)$.
Then the derivative $f'(x)=x{1\over \ln2}{1\over x}+\log x+(m-x)
{1\over \ln2}{-1\over m-x}-\log (m-x)=\log x-\log(m-x)$.
Let $f'(x)=0$. Then $\log x=\log(m-x)$, which implies $x={m\over2}$.
So $f(x)$ achieves the minimum when $x={m\over2}$. To continue,
\begin{eqnarray*}
D_m&\ge&m+\min_{1\le i\le m-1}\{i\log i+(m-i)\log (m-i)\}\\
&\ge&m+{m\over2}\log{m\over2}+{m\over2}\log{m\over2}\\
&=&m+m\log{m\over2}\\
&=&m\log m
\end{eqnarray*}

\item Theorem: Any comparison-based sorting algorithm requires 
$\Omega(n\log n)$ time in the average case.

{\em Proof}\quad Let $T$ be a decision tree for any algorithm that
sorts $n$ elements by comparisons. $T$ has $n!$ leaves. So the
average depth of the leaves is ${1\over n!}\sum_id_i\ge {1\over n!}D_{n!}
\ge {1\over n!}n!\log n!=\log n!$.
So for any sorting algorithm, the average-case
time complexity is the average depth of the decision tree, thus at least $\log n!
=\Omega(n\log n)$.

\item Bucket Sort: not comparison-based, with $O(n)$ average time.

Let $A=(a_1,\ldots,a_n)$, where $a_i\in[0,1)$ is generated randomly 
such that $a_1,\ldots,a_n$ are distributed uniformly in $[0,1)$.
For example, $A=(0.78,0.17,0.39,0.26,0.72,0.94,0.61)$. 
Let $B$ be an array of intervals in $[0,1)$, where the length of
each interval is $1\over n$. In particular, $B[i]$ represents
interval $[{i\over n},{i+1\over n})$ for $0\le i\le n-1$. Each
$B[i]$ will be a pointer to a list of numbers falling into the
corresponding interval.

\qquad for $i\leftarrow 1$ to $n$ 

\qquad\qquad insert $A[i]$ into list $B[\lfloor nA[i]\rfloor]$

\qquad for $i\leftarrow 0$ to $n-1$

\qquad\qquad sort list $B[i]$ using any sorting algorithm

\qquad concatenate $B[0],\ldots,B[n-1]$ into a list for output

Why should $A[i]$ go to list $B[\lfloor nA[i]\rfloor]$? Assume 
$A[i]$ goes to $B[x]$. Then ${x\over n}\le A[i]<{x+1\over n}$.
So $x\le nA[i]<x+1$. So $nA[i]-1<x\le nA[i]$. Since $x$ is an
integer, $x=\lfloor nA[i]\rfloor$.

Time complexity: $\Theta(n^2)$ or $\Theta(n\log n)$ in the worst case;
$\Theta(n)$ in the average case since each list in $B$ has 1 element
on average.
 
\end{itemize}

\subsection{The adversary method}

\begin{itemize}

\item The adversary strategy: An adversary provides worst-case
input that forces all algorithms solving the problem to work as
hard as possible.

Example: Alice (1) has a date (month/day) in mind (this is what she
tells Bob); (2) answers Bob's questions honestly; and
(3) wants to force Bob to ask as many questions as possible.
Bob (1) asks yes/no questions and
(2) wants to guess the date as quickly as possible.

Bob: Is it in the winter? (December, January and February)

Alice: No (since there are more dates in the other three seasons).

Bob: Is the first letter of the month's name in the first half of 
the alphabet?

Alice: Yes (since among the remaining 9 months, 6 are in the
first half of the alphabet, March, April, May, June, July,
August).

$\cdots\cdots$

Alice didn't pick a date at all. Is this cheating? Not to Bob.
In fact, Alice will not do so until the need for consistency
in her answers pins her down. This is how a good adversary should
be.

For a comparison-based model, the adversary works as follows:

\vskip 0.25cm
\begin{center}
\includegraphics[height=1.3in]{f15.pdf}
\end{center}

\item Finding the max and min at the same time:

Let $C_{1,n}(n)$ be the number of comparisons necessary and
sufficient to find the maximum and minimum of $n$ integers.
Prove that $C_{1,n}(n)=\lceil{3n\over2}\rceil-2$.

Upper bound: $C_{1,n}(n)\le\lceil{3n\over2}\rceil-2$. Wish
to design an algorithm which takes at most $\lceil{3n\over2}\rceil-2$
comparisons. The algorithm first uses $n-1$ comparisons to find the
maximum among the $n$ numbers and then $\lceil{n\over2}\rceil-1$ 
more comparisons to find the minimum among the $\lceil{n\over2}\rceil$
non-winners in the first round.

\vskip 0.25cm
\begin{center}
\includegraphics[height=1.1in]{f16.pdf}
\end{center}

Lower bound: $C_{1,n}(n)\ge\lceil{3n\over2}\rceil-2$.
Assume all keys are distinct. At any time in any algorithm,
there are four kinds of elements.

Novice: not compared yet;

Winner: always wins so far;

Loser: always loses so far;

Moderate: wins some and loses some.

Define a state $(i,j,k,l)$, where $i$, $j$, $k$, and $l$ are
the numbers of novices, winners, losers, and moderates, respectively.
The initial state is $(n,0,0,0)$ and the final state is $(0,1,1,n-2)$.
Question: What is the minimum number of comparisons from the initial
state to the final state? A state transition table is given below:

\begin{center}
\begin{tabular}{r|l}
& $(i,j,k,l)$\\\hline
NN & $(i-2,j+1,k+1,l)$\\
NW & $(i-1,j,k,l+1)$ or $(i-1,j,k+1,l)$\\
NL & $(i-1,j+1,k,l)$ or $(i-1,j,k,l+1)$\\
NM & $---$\\
WW & $(i,j-1,k,l+1)$\\
WL & $(i,j,k,l)$ or $(i,j-1,k-1,l+2)$\\
WM & $---$\\
LL & $(i,j,k-1,l+1)$\\
LM & $---$\\
MM & $---$
\end{tabular}
\end{center}

Assume that an adversary provides the input and uses the following
strategy:

\begin{center}
\begin{tabular}{r|c|c|c|c|c|c}
algorithm & NN & NW & NL & WW & WL & LL \\\hline
adversary & arbitrary & W wins & N wins & arbitrary & W wins & arbitrary
\end{tabular}
\end{center}

Why is the adversary able to make the choices it does in the table
above? For an NW comparison, it can make N to be arbitrarily small 
so that W is the winner. For the similar reason in an NL comparison,
it can make N to be arbitrarily large so that L is the loser.
In a WL comparison, the adversary can increase W (or decrease L) so
that W beats L without changing the outcomes of the previous
comparisons. Since the algorithm cannot see the numbers but only
the comparison results, this action by the adversary is totally
legal. 

In the initial state, $i+j+k=n$, while in the final state 
$i+j+k=2$. Since only a WW or LL comparison decreases $i+j+k$
(by 1), we need to make at least $n-2$ WW or LL comparisons.
Next consider the change of $i$ in the initial and final states.
It decreases from $i=n$ to $i=0$. Since WW and LL comparisons
never change the value of $i$, some NN, NW, NL comparisons are
needed. Since such a comparison decreases $i$ by at most $2$,
a minimum of $\lceil{n\over2}\rceil$ comparisons are needed.
So the total number of comparisons necessary is
$n-2+\lceil{n\over2}\rceil=\lceil{3n\over2}\rceil-2$.
 
\end{itemize}

\newpage

\subsection{Reduction}

\begin{itemize}

\item Proving lower bounds by reduction:

Let $P$ be the problem whose lower bound, $f(n)$, we wish to prove.
Let $Q$ be a problem whose lower bound, $g(n)$, is known. For any 
algorithm $A$ for $P$, we try to define an algorithm $B$ for $Q$ which
calls $A$ as follows:

\vskip 0.25cm
\begin{center}
\includegraphics[height=1.8in]{f17.pdf}
\end{center}

We assume that there is an algorithm, say $A$, for $P$ that takes
time $o(f(n))$. Then from the chart above, there is an algorithm,
$B$, for $Q$, with time $t_1(n)+o(f(n))+t_2(n)$. If $t_1(n)+o(f(n))+t_2(n)
=o(g(n))$, then we just found an algorithm which beats the proven
lower bound, $g(n)$, for $Q$. This is of course impossible. 
So there is no algorithm for $P$ with time faster than $f(n)$.
Therefore, $f(n)$ is a lower bound for $P$.

\item The convex hull problem: Given a set $P$ of $n$ points on the 2-D plane.
The {\em convex hull} of the point set, denoted by $CH(P)$, is a convex
polygon, represented by its vertices in counterclockwise order, say,
$p_1, p_2, \ldots, p_k$, where $p_i\in P$ for $i=1, 2, \ldots, k$,
such that the polygon contains all the other points in $P$. See the
following figure for an example.

\vskip 0.25cm
\begin{center}
\includegraphics[height=1.2in]{f18.pdf}
\end{center}

Next we will prove that any algorithm that finds $CH(P)$ for a given
$P$ of $n$ points takes at least $n\log n$ time in the worst case.

Assume there is an algorithm $A$ for the convex hull problem which
takes $o(n\log n)$ worst-case time. Now we will design a sorting
algorithm $B$ that calls $A$. For any input list of $n$ numbers
$x_1, x_2, \ldots, x_n$, algorithm $B$ first converts the list in $O(n)$
to a point set $(x_1, x_1^2), (x_2, x_2^2), \ldots, (x_n, x_n^2)$
and feed the point set as input to algorithm $A$. Algorithm $A$ then
computes the convex hull of the point set in $o(n \log n)$ time.
Because of the special locations of the points, the convex hull will
contain all points as the vertices of the polygon, and the points
are output in counterclockwise order. Algorithm $B$ then spends $O(n)$
to search the point in the output list with the smallest $x$-coordinate,
and from there to produce a list of $n$ $x$-coordinates, which is clearly
the sorted version of the original input list to $B$. So algorithm $B$
sorts a list of $n$ numbers in time $O(n)+o(n\log n)+O(n)=o(n\log n)$.
This is a contradiction to the $\Omega(n\log n)$ lower bound for sorting.

\vskip 0.25cm
\begin{center}
\includegraphics[height=1.6in]{f19.pdf}
\end{center}

\end{itemize}

\subsection{Trivial lower bounds by input/output}

\begin{itemize}

\item We can also use the amount of input data that any correct algorithm needs to process
or the amount of data that any  correct algorithm needs to produce to establish
trivial lower bounds.

\item For example in the matrix multiplication problem, since the
multiplication of two $n\times n$ matrices is another $n\times n$
matrix and there are $n^2$ entries to compute, $n^2$ is obviously
a lower bound. (Not yet proved to be tight.)

\item For another example, consider the problem of generating all permutations
of $n$ objects. Since there are $n!$ permutations to generate, $\Omega(n\cdot n!)$ is 
an obvious lower bound. (Tight.)

\item Evaluation of an $n$-degree polynomial requires that all $n+1$ coefficients be
processed. Thus the trivial lower bound is $\Omega(n)$. (Tight.)

\item Some bounds established by input/output are not correct lower bounds. 
For example, searching target in a sorted array of $n$ elements does not require to check 
all $n$ elements.

\item Some lower bounds established by input/output are not trivial to prove.
For example, determining connectivity of an undirected graph by its adjacency matrix
requires to check the existence of each of the $n(n-1)/2$ potential edges.

\end{itemize}

\newpage

\section{NP-Completeness}

\subsection{Optimization versus decision}

\begin{itemize}

\item Example: Graph Coloring

Optimization: Given $G=(V,E)$, color the nodes with the minimum
number of colors such that no two adjacent nodes connected by an
edge have the same color.

Decision: Given $G=(V,E)$ and $B>0$ (an integer), is there a
feasible coloring of the nodes using at most $B$ colors?

\begin{center}
\begin{tabular}{c|c}
Optimization & Decision\\\hline
Minimize $f(S)$ & $f(S)\le B$\\\hline
Maximize $f(S)$ & $f(S)\ge B$
\end{tabular}
\end{center}

\item Theorem: If there is an algorithm with time $\Theta(T(n))$ for
OPT, then there is an algorithm with time $\Theta(T(n))$ for DEC.
In other words, DEC is always no harder than its OPT.

\vskip 0.25cm
\begin{center}
\includegraphics[height=2in]{f28.pdf}
\end{center}

\end{itemize}

\subsection{The class {\bf P}}

\begin{itemize}

\item Definition: {\bf P} is the class of problems solvable in
polynomial time. 

$O(n^c)$, where $c$ is a constant.

Tractable (not so hard).

\item Why use polynomial as the criterion?

If a problem is not in {\bf P}, it will be extremely expensive
and probably impossible to solve in practice for large sizes.

Polynomials have nice closure properties: $+, -, *$, and composition.

{\bf P} is independent of models of computation: $1$TM, $k$TM, RAM, etc..

\end{itemize}

\subsection{The class {\bf NP}}

\begin{itemize}

\item Definition: {\bf NP} is the class of problems solvable in 
polynomial time by nondeterministic algorithms.

\item Definition: Nondeterministic algorithms$\Leftrightarrow$
nondeterministic computer(a hypothetical model that doesn't exist)

\begin{itemize}

\item Guessing phase: Guess a solution (always on target).

\item Verifying phase: Verify the solution.

\end{itemize}

\item Example: Graph Coloring is in {\bf NP}. Consider the decision
problem. A nondeterministic algorithm first guesses a coloring scheme
(in polynomial time) and then verify if the coloring uses no more
than $B$ colors and if any two adjacent nodes have different colors
(in polynomial time). 

\item Theorem: {\bf P}$\subseteq${\bf NP}.

Any ordinary (deterministic) algorithm is a special case of
a nondeterministic algorithm.

\item Theorem: Any $\Pi\in${\bf NP} can be solved by a deterministic
algorithm in time $O(c^{p(n)})$ for some $c>0$ and polynomial $p(n)$.

\item Open problem: {\bf NP}$\subseteq${\bf P}? or
{\bf P}$=${\bf NP}? 

\end{itemize}

\subsection{Polynomial reduction}

\begin{itemize}

\item Definition: Let $\Pi_1$ and $\Pi_2$ be two decision problems,
and $\{I_1\}$ and $\{I_2\}$ be sets of instances for $\Pi_1$ and
$\Pi_2$, respectively. We say there is a polynomial reduction
from $\Pi_1$ to $\Pi_2$, or $\Pi_1\propto_p\Pi_2$ if there is
$f:\{I_1\}\rightarrow$ subset of $\{I_2\}$ such that (1)
$f$ can be computed in polynomial time and (2) $I_1$ has a 
``yes'' solution if and only if $f(I_1)$ has a ``yes''
solution. 

\item Theorem: If $\Pi_1\propto_p\Pi_2$, then $\Pi_2\in${\bf P}
implies $\Pi_1\in${\bf P}.

\item Theorem: If $\Pi_1\propto_p\Pi_2$ and $\Pi_2\propto_p\Pi_3$,
then $\Pi_1\propto_p\Pi_3$.

\item {\em Remark:} $\propto_p$ means ``no harder than''.

\item Example: PARTITION$\propto_p$KNAPSACK.

PARTITION:

INSTANCE: A finite set $A$ of numbers.

QUESTION: Is there $A'\subseteq A$ such that $\sum_{a\in A'}a=
\sum_{a\in A-A'}a$?

KNAPSACK:

INSTANCE: $U=\{u_1,\ldots,u_n\}$, $w:U\rightarrow Z^+$,
$v:U\rightarrow Z^+$, and $W,V\in Z^+$.

QUESTION: Is there $U'\in U$ with $\sum_{u\in U'}w(u)\le W$
such that $\sum_{u\in U'}v(u)\ge V$?

For any instance for PARTITION: $A=\{a_1,\ldots,a_n\}$,
let $U=\{u_1,\ldots,u_n\}$ with each $u_i$ corresponding to
each $a_i$. Let $w(u_i)=v(u_i)=a_i$, and $W=V={1\over2}\sum_i a_i$.

This reduction can certainly be done in polynomial time. Next we
need to prove that there is $A'\subseteq A$ with $\sum_{a\in A'}a
={1\over2}\sum_i a_i$ if and only if there is $U'\subseteq U$
with $\sum_{u\in U'}w(u)\le W$ and $\sum_{u\in U'}v(u)\ge V$.
The proof is straightforward.

\end{itemize}

\subsection{The class {\bf NPC}}

\begin{itemize}

\item Definition: {\bf NPC} ({\bf NP}-complete) is the class of
the hardest problems in {\bf NP}, or equivalently, $\Pi\in${\bf NPC}
if $\Pi\in${\bf NP} and $\forall\Pi'\in${\bf NP}, $\Pi'\propto_p\Pi$.

\item Theorem: If $\Pi_1\in${\bf NPC}, $\Pi_2\in${\bf NP}, and
$\Pi_1\propto_p\Pi_2$, then $\Pi_2\in${\bf NPC}.

\item Theorem: If $\exists\Pi\in${\bf NPC} such that $\Pi\in${\bf P},
then {\bf P}$=${\bf NP}.

\item Theorem: If $\exists\Pi\in${\bf NPC} such that $\Pi\not\in$
{\bf P}, then {\bf P}$\not=${\bf NP}.
 
\item Satisfiability (SAT):

INSTANCE: A boolean formula $\alpha$ in $CNF$ with variables 
$x_1,\ldots,x_n$ (e.g., $\alpha=(x_1\lor x_2)\land(\neg x_2\lor x_3)$).

QUESTION: Is $\alpha$ satisfiable? (Is there a truth assignment to
$x_1,\ldots,x_n$ such that $\alpha$ is true?)

Cook's Theorem: SAT$\in${\bf NPC}. (Need to prove (1) SAT$\in${\bf NP}
and (2) $\forall\Pi\in${\bf NP}, $\Pi\propto_p$SAT.)

\end{itemize}

\subsection{NP-complete problems}

\begin{itemize}

\item How to prove $\Pi$ is {\bf NP}-complete:

\begin{itemize}

\item Show that $\Pi\in${\bf NP}.

\item Choose a known {\bf NP}-complete $\Pi'$.

\item Construct a reduction $f$ from $\Pi'$ to $\Pi$.

\item Prove that $f$ is a polynomial reduction by showing that
(1) $f$ can be computed in polynomial time and (2) $\forall I'$ for
$\Pi'$, $I'$ is a yes-instance for $\Pi'$ if and only if $f(I')$ is
a yes-instance for $\Pi$.

\end{itemize}

\item Seven basic {\bf NP}-complete problems.

\vskip 0.25cm
\begin{center}
\includegraphics[height=1.3in]{f29.pdf}
\end{center}

\begin{itemize}

\item 3-Satisfiability (3SAT):

INSTANCE: A formula $\alpha$ in CNF with each clause having
three literals.

QUESTION: Is $\alpha$ satisfiable?

\item 3-Dimensional Matching (3DM):

INSTANCE: $M\subseteq X\times Y\times Z$, where $X,Y,Z$ are disjoint and 
of the same size.

QUESTION: Does $M$ contain a matching, which is $M'\subseteq M$ with
$|M'|=|X|$ such that no two triples in $M'$ agree in any coordinate?

\item PARTITION:

INSTANCE: A finite set $A$ of numbers.

QUESTION: Is there $A'\subseteq A$ such that $\sum_{a\in A'}a=
\sum_{a\in A-A'}a$?

\item Vertex Cover (VC):

INSTANCE: A graph $G=(V,E)$ and $0\le k\le |V|$.

QUESTION: Is there a vertex cover of size $\le k$, where a vertex
cover is $V'\subseteq V$ such that $\forall (u,v)\in E$, either
$u\in V'$ or $v\in V'$?

\item Hamiltonian Circuit (HC):

INSTANCE: A graph $G=(V,E)$.

QUESTION: Does $G$ have a Hamiltonian circuit, i.e., a tour that
passes through each vertex exactly once?

\item CLIQUE:

INSTANCE: A graph $G=(V,E)$ and $0\le k\le |V|$.

QUESTION: Does $G$ contain a clique (complete subgraph) of size $\ge k$?

\end{itemize}

\end{itemize}

\newpage

\subsection{Approximation algorithms}

\begin{itemize}
\item Goal: A fast algorithm (polynomial time such as $O(n)$ and
$O(n\log n)$) that produces near-optimal solution.

\item Notation:

--- $\Pi$: An optimization {\bf NP}-complete problem.
                                                                                
--- $I$: Any instance of the problem.
                                                              
--- $OPT$: The exponential-time optimal algorithm.
                                                                                
--- $A$: A polynomial-time approximation algorithm.
                                                                                
--- $OPT(I)$: Optimal solution (value) for $I$ obtained by applying $OPT$.

--- $A(I)$: Approximation solution (value) for $I$ when $A$ is used.
	
\item How good is $A$?
                                                                          
--- Performance ratio:
                                                                               
\qquad $R_A^n=\max_{\forall I: |I|=n}\{A(I)/OPT(I)\}$ for minimization problems.

\qquad $R_A^n=\max_{\forall I: |I|=n}\{OPT(I)/A(I)\}$ for maximization problems.

\qquad Asymptotic ratio: $R_A^\infty=\lim_{n\to\infty}\sup R_A^n$.
                                                                                
\qquad $R_A^n, R_A^\infty>1$. The smaller the better.
                                                                               
--- Relative error:

\qquad $\epsilon_A^n=\max_{\forall I: |I|=n}\{|A(I)-OPT(I)|/OPT(I)\}$.
	
\item SET COVERING
                                                               
INSTANCE: A finite set $X$ and a family $F$ of subsets of $X$ such that
$X=\cup_{S\in F}S$.
                                                                               
GOAL: Determine $C\subseteq F$ with the minimum size such that
$X=\cup_{S\in C}S$, i.e., members of $C$ cover all of $X$.
                                                                              
The decision version of SET COVERING is {\bf NP}-complete.
(Reduce from VERTEX COVER.)
                                                                            
An example of SET COVERING:
$X=\{1,2,3,4,5,6,7,8,9,10,11,12\}$ and $F=\{S_1,S_2,S_3,S_4,S_5,S_6\}$,
where $S_1=\{1,2,5,6,9,10\}$, $S_2=\{6,7,10,11\}$,
$S_3=\{1,2,3,4\}$, $S_4=\{3,5,6,7,8\}$, $S_5=\{9,10,11,12\}$, and
$S_6=\{4,8\}$. The optimal solution is $C=\{S_3,S_4,S_5\}$.

An approximation algorithm based on the greedy strategy:
	
GREEDY-SET-COVER($X,F$)
                                                                                
\qquad $U\leftarrow X$
                                                                               
\qquad $C\leftarrow\emptyset$
                                                                           
\qquad while $U\not=\emptyset$ do
                                                                             
\qquad\quad select an $S\in F$ that maximizes $|S\cap U|$
                                                                         
\qquad\quad $U\leftarrow U-S$
                                                                          
\qquad\quad $C\leftarrow C\cup \{S\}$
                                                                           
\qquad return $C$
                                                                          
Applying GREEDY-SET-COVER to the example, we have a cover $C$ of
size $4$ containing $S_1,S_4,S_5,S_3$.
	
Theorem: For any instance, GREEDY-SET-COVER finds a cover $C$ of size
no larger than $H(\max\{|S|:S\in F\})$ times the optimal size. (Note:
$H(d)=\sum_{i=1}^d1/i$ is the $d$th harmonic number.)

{\em Proof.}\quad Let $S_i$ be the $i$th subset selected by GREEDY-SET-COVER.
We assume that the algorithm incurs a cost of $1$ when it adds $S_i$ to $C$.
We spread this cost of selecting $S_i$ evenly among the elements covered for 
the first time by $S_i$. 
Let $c_x$ denote the cost allocated to element $x\in X$.
Each element is assigned a cost only once, when it is covered for the first
time. If $x$ is covered for the first time by $S_i$, then
$$c_x={1\over |S_i-(S_1\cup S_2\cup\cdots\cup S_{i-1})|}.$$
The algorithm finds a solution $C$ of total cost $|C|$, and this cost has
been spread out over the elements of $X$. Therefore, since the optimal
cover $C^*$ also covers $X$, we have
$$|C|=\sum_{x\in X}c_x\le\sum_{S\in C^*}\sum_{x\in S}c_x.$$
The remainder of the proof rests on the following key inequality.
For any $S\in F$,
$$\sum_{x\in S}c_x\le H(|S|).$$
Suppose the above is true. We have
$$|C|\le \sum_{S\in C^*}\sum_{x\in S}c_x\le \sum_{S\in C^*}H(S)
\le |C^*|\cdot H(\max\{|S|:S\in F\}).$$
Now let us focus on the proof of the inequality
$\sum_{x\in S}c_x\le H(|S|)$. For any $S\in F$ and $i=1,\ldots,|C|$,
let
$$u_i=|S-(S_1\cup S_2\cup\cdots\cup S_i)|$$
be the number of elements in $S$ remaining uncovered after
$S_1,\ldots,S_i$ have been selected by the algorithm. We define
$u_0=|S|$ to be the number of elements in $S$, which are initially
uncovered. Let $k$ be the least index such that $u_k=0$, so that
every element in $S$ is covered by at least one of the
sets $S_1,\ldots,S_k$. Then $u_{i-1}\ge u_i$ and $u_{i-1}-u_i$
elements of $S$ are covered for the first time by $S_i$ for
$i=1,\ldots,k$. Thus
$$\sum_{x\in S}c_x=\sum_{i=1}^k(u_{i-1}-u_i)
{1\over |S_i-(S_1\cup\cdots\cup S_{i-1})|}.$$
Observe that
$$|S_i-(S_1\cup\cdots\cup S_{i-1})|\ge|S-(S_1\cup\cdots\cup S_{i-1})|
=u_{i-1},$$
because the greedy choice of $S_i$ guarantees that $S$ cannot cover more
new elements than $S_i$ does. So we obtain
$$\sum_{x\in S}c_x\le\sum_{i=1}^k(u_{i-1}-u_i){1\over u_{i-1}}.$$
For integer $a$ and $b$ with $a<b$,
$$H(b)-H(a)=\sum_{i=a+1}^b{1\over i}\ge (b-a){1\over b}.$$
Using this inequality,
$$\sum_{x\in S}c_x\le\sum_{i=1}^k(H(u_{i-1})-H(u_i))
=H(u_0)-H(u_k)=H(u_0)=H(|S|).$$
                      
Corollary: GREEDY-SET-COVER has a performance ratio of
$\ln|X|+1$.

\end{itemize}

\newpage

\section{Parallel Algorithms}

\subsection{Introduction}

\begin{itemize}

\item The PRAM model: 

Parallel Random Access Machine

\vskip 0.25cm
\begin{center}
\includegraphics[height=1.4in]{f20.pdf}
\end{center}

Processors $P_0,\ldots,P_{p-1}$: (1) working synchronously (at the same pace);
(2) communicating through the shared memory; (3) $P_i=$RAM, $\forall i$;
(4) unit-time memory access; (5) SIMD: Single-instruction, multiple-data;
(6) MIMD: Multiple-instruction, multiple data.

\item Concurrent versus exclusive memory accesses:
EREW, CREW, ERCW, CRCW.

How to implement concurrent write: (1) common; (2) arbitrary;
(3) priority; (4) combining.

\item Pseudocode:

``for all $x\in X$ in parallel do $instruction(x)$''

Statement includes the following steps:

(1) Assign a processor to each $x\in X$. (In constant time,
$P_{code(x)}\Leftrightarrow x$ and $P_i\Leftrightarrow x=code^{-1}(i)$.)

(2) Execute in parallel and by the assigned processors all those
operations specified by $instruction(x)$. (The central control unit
is used to synchronize.)

Example: Find the minimum of $n$ numbers.

$A$ is the array of $n$ numbers and $B$ is an auxiliary array
(if $B[i]=1$ then $A[i]$ is not the min, but if $B[i]=0$, then
$A[i]$ is the min). 
$min$ stores the output.

\qquad 1. for all $i$, $1\le i\le n$, in parallel do $B[i]\leftarrow 0$

\qquad 2. for all ordered pairs $(i,j)$, $1\le i,j\le n$, in parallel do

\qquad\qquad if $A[i]<A[j]$ $B[j]\leftarrow 1$

\qquad 3. for all $i$, $1\le i\le n$, in parallel do

\qquad\qquad if $B[i]=0$ $min\leftarrow A[i]$

Analysis: CRCW algorithm; input size is $n$; the number of processors
is $p=n^2$; the time is $t=\Theta(1)$. So the algorithm finds the
minimum of $n$ numbers in constant time, using $n^2$ processors.

\item Complexity of parallel algorithms: 

$n$: Input size;

$p$: The number of processors;

$t$: The parallel running time;

polylog: $O((\log n)^k)$ for some constant $k$;

A parallel algorithm is efficient if it takes polylog time
$O((\log n)^k)$ using a polynomial $O(n^c)$ number of processors;

The class {\bf NC}: Nick (Pippenger)'s class, includes problems
with efficient parallel algorithms;

A parallel algorithm is optimal if $pt=O(n)$;

{\bf P}-complete problems: in {\bf P} and no efficient parallel algorithms 
have been found so far; 

A hypothetical picture based on {\bf P}$\not=${\bf NP} and
{\bf P}$\not=${\bf NC}:

\vskip 0.25cm
\begin{center}
\includegraphics[height=1in]{f21.pdf}
\end{center}

\item Reducing the number of processors (Brent's theorem):

If a $p$-processor PRAM algorithm $A$ runs in parallel time
$t$, then for any $p'<p$ there is a $p'$-processor PRAM 
algorithm $A'$ for the same problem that runs in parallel
time $O(pt/p')$. (Note: $pt=p't'$.)

\end{itemize}

\subsection{Algorithmic technique 1: Pointer jumping}

\begin{itemize}

\item Manipulating lists with pointers

\item Example 1. List ranking.

$L$ is a list of $n$ elements, where $next[i]$ points to the next element
following $i$ in $L$. Output $rank[i]$, $\forall i\in L$, is the rank/order
number of $i$ in $L$. $rank[i]=0$ if $next[i]=nil$ and
$rank[i]=rank[next[i]]+1$ if $next[i]\not=nil$.

Assign a process to each element in $L$.

Each node in the list has two fields: $rank$ and $next$.

\vskip 0.25cm
\begin{center}
\includegraphics[height=1.3in]{f22.pdf}
\end{center}

$ListRanking(L)$

\qquad for all $i\in L$ in parallel do

\qquad\qquad if $next[i]=nil$ $rank[i]\leftarrow 0$

\qquad\qquad else $rank[i]\leftarrow 1$

\qquad while $\exists i$ such that $next[i]\not=nil$ do

\qquad\qquad for all $i\in L$ in parallel do

\qquad\qquad\qquad if $next[i]\not=nil$

\qquad\qquad\qquad\qquad $rank[i]\leftarrow rank[next[i]]+rank[i]$

\qquad\qquad\qquad\qquad $next[i]\leftarrow next[next[i]]$

Analysis: $p=n$ and $t=O(\log n)$.

\item Example 2. Prefix computation.

Input: $x_1,x_2,\ldots,x_n$.

Output: $y_1,y_2,\ldots,y_n$ such that $y_1=x_1$ and 
$y_i=y_{i-1}\otimes x_i=x_1\otimes\cdots\otimes x_i$, where
$\otimes$ is any associative binary operator.

Define $[i,j]=x_i\otimes\cdots\otimes x_j$, $1\le i\le j\le n$.
Then $[i,i]=x_i$ and $[i,j]=[i,k]\otimes [k+1,j]$, $1\le i\le k<j\le n$.

We wish to compute $[1,i]$, $1\le i\le n$.

\vskip 0.25cm
\begin{center}
\includegraphics[height=1.3in]{f23.pdf}
\end{center}

$ListPrefix(L)$

\qquad for all $i\in L$ in parallel do

\qquad\qquad $y[i]\leftarrow x[i]$

\qquad while $\exists i\in L$ such that $next[i]\not=nil$ do

\qquad\qquad for all $i\in L$ in parallel do

\qquad\qquad\qquad if $next[i]\not=nil$

\qquad\qquad\qquad\qquad $y[next[i]]\leftarrow y[i]\otimes y[next[i]]$

\qquad\qquad\qquad\qquad $next[i]\leftarrow next[next[i]]$

\item Example 3. Computing the depth of each node in an $n$-node
binary tree.

\vskip 0.25cm
\begin{center}
\includegraphics[height=1.9in]{f24.pdf}
\end{center}
 
The algorithm contains two steps. One is to construct the so-called
Euler tour. The second step is to do a prefix computation, which
requires $3n$ processors in $O(\log n)$ parallel time. 

How to construct an Euler tour in $O(1)$ parallel time with $n$ processors:

\qquad for all nodes $i$ in the tree in parallel do

\qquad\qquad if left child $j$ exists

\qquad\qquad\qquad make $i.A$ to point to $j.A$

\qquad\qquad\qquad make $j.C$ to point to $i.B$

\qquad\qquad else make $i.A$ to point to $i.B$

\qquad\qquad if right child $k$ exists

\qquad\qquad\qquad make $i.B$ to point to $k.A$

\qquad\qquad\qquad make $k.C$ to point to $i.C$

\qquad\qquad else make $i.B$ to point to $i.C$

\vskip 0.25cm
\begin{center}
\includegraphics[height=4in]{f25.pdf}
\end{center}

\end{itemize}

\subsection{Algorithmic technique 2: Balanced binary tree}

\begin{itemize}

\item Each internal node represents the computation of a subproblem.

\item The root represents the computation of the overall problem.

\item Bottom-up design.

\item Parallel time is at least the height of the tree.

\item Example: Computing the minimum of $n$ numbers.

\vskip 0.25cm
\begin{center}
\includegraphics[height=1.1in]{f26.pdf}
\end{center}

\item $p={n\over2}$ and $t=\log n$.

\end{itemize}

\subsection{Algorithmic technique 3: Divide and conquer}

\begin{itemize}

\item Top-down design with subproblems solved in parallel.

\item Parallel time is at least the depth of the recursion.

\item Example: Given a polynomial $p(x)$ of degree $n=2^k-1$.
Evaluate $p(x)$ at $x=x_0$. 

$p(x)=x^{n+1\over2}q(x)+r(x)$, where $deg(q)=2^{k-1}-1$
and $deg(r)\le 2^{k-1}-1$. 

\vskip 0.25cm
\begin{center}
\includegraphics[height=2.3in]{f27.pdf}
\end{center}

Depth is $O(\log n)$. So is the parallel time.

\end{itemize}

\subsection{Algorithmic technique 4: Compression and collapsing}

\begin{itemize}

\item Example: Finding the minimum.

$A[i]$ and $A[i+1]$ is compressed to $\min\{A[i],A[i+1]\}$ for odd $i$.

$[3,7,8,3,9,2,3,1]\Rightarrow [3,3,2,1]\Rightarrow [3,1]
\Rightarrow [1]$.

\item Example: Graph algorithms: Compress a graph recursively into
a single (super-)vertex.

\item Example: Euler tour: Collapse a node into three subnodes.

\end{itemize}

\end{document}

