% ---- generic definition ----
UnionFind for Disjoint Sets: given \textit{n} singletons: 
$S_i = {i}$ for $i = 1, ..., n$
% operations
1. \textit{union}($S_i$, $S_j$) return $S_i U S_j$, where $S_i$ and $S_j$ are disjoint; 
2. \textit{findSet(i)} gives the set of \textit{i}.
3. \textit{makeSet(i)} creates a singleton set with \textit{i}
% generic definition from above can be omitted
\\
%
% these two heuristics improve running time
Disjoint set forests: set is a rooted tree; each node contains a member; 
root acts as the set representative
1. union (by rank): rank is (upper bound) height of node 
(\# of edges from leaf to node); make root with smaller rank point to root with larger rank
2. findSet (by path compression): first pass finds the path to the root; 
second pass updates all the nodes in the path to point to root

% complexity
(amortized) $T(n) = O(m\alpha(n))$, where m = total operations, n = \# of elements, 
$\alpha(n) (\le 4)$ is a very slowly growing fun (inverse Ackermann's fun)

\textbf{Some properties of ranks}\\
Let $\sigma'$ be $\sigma$ with all finds removed. Let $Forest(\sigma')$
be the forest resulting from executing $\sigma'$.

P1: Any tree has no larger than $q+1$ nodes in $Forest(\sigma')$. (Branches).

P2: Node $i$ has at least $2^{r(i)-1}$ descendants (including $i$)
in $Forest(\sigma')$. (Induct on $r(i)$. Consider the subtree rooted
at node $i$, which must be constructed by some union operations. There
must be a union operation that 
merges a smaller-size tree with height $r(i)-1$
to a tree rooted at $i$. Thus the number of nodes in the tree
obtained after the union is at least $2\cdot 2^{r(i)-2}=2^{r(i)-1}$.)

P3: $r(i)\le \lfloor\log (q+1)\rfloor+1$, for any $i$. 
(By contradiction, assume $r(i)\ge \lfloor\log (q+1)\rfloor+2$.
By P2, the number of descendant is at least $2^{\lfloor\log (q+1)\rfloor+1}
>2^{\log (q+1)}=q+1$. A contradiction to P1.)

P4: There are at most ${n\over 2^{r-1}}$ nodes of rank $r$. 
(By contradiction, assume for some rank $r$ there are more than 
${n\over 2^{r-1}}$ nodes with rank $r$. By P2 each node has at least $2^{r-1}$
descendants. The number of nodes in these (disjoint) subtrees is more than
${n\over 2^{r-1}}\cdot 2^{r-1}=n$. A contradiction!)

P5: At any time during the execution of $\sigma$, the ranks of the nodes
on a path from a leaf to a root increase strict monotonically.
(For nodes $w$ and $v$, if $w$ is a proper descendant of $v$ at some time of executing $\sigma$,
then they must still have the same descendant-ancester relation in $Forest(\sigma')$. So $r(w)<r(v)$.)

1. $F(0) = 1$, $F(n) = 2^{F(n-1)}$ for $n \ge 1$. $G$ is inversion of $F$. \\
2. $G(n) = min\{k \ge 0 | F(k) \ge n\}$; $F$ grows extremely fast; $G(n)$ measures iterations needed for $F(k)$ to surpass or match $n$ 
3. $G(F(n)) = n$ and $G(n) = \log^*(n)$\\
4. Ranks are distributed into groups; $G(0) = {0,1}$, $G(1) = {2}$, $G(2) = {3,4}$...\\
5. Total cost of executing $\sigma$ with $m=q+p$ operations is:  
a) cost of $q$'s $union$ b) cost of $p$'s $find$; $T(n) = O(q+pG(n)+nG(n))$ (pessimistic) = $O(mlog^*(n))$ if $m=\Theta(n)$ (tight bound by amortized analysis)


% including notes on two heaps for completeness - although I don't think we did
% problems on heaps or studied it in class?
% binary heaps
\textbf{Binary Heaps:} 
1. max heap property - parent is no smaller(or $\ge$) than its children;
2. as an array ($H[n]$) - for $H[i]$ , parent = $H[\lfloor i/2 \rfloor]$, 
left child = $H[2i]$, right child = $H[2i+1]$
3. heap sort (worst-case): $O(nlogn)$
4. heap height: $\Omega(logn)$
5. operations: $Insert(H, x)-log(n)$, $ExtractMax(H)-log(n)$, 
$FindMax(H)-O(1)$, $MakeHeap(H)-O(n)$
% useful to understand array ordering -- homework 2 problem 4
% 5. \href{https://www.mimuw.edu.pl/~erykk/algovis/heapsort.html}{Heap Sort Viz.}

% fibonacci heaps
\textbf{Fibonacci Heaps/Mergeable Heaps: }
1. implemented using binary heaps
2. new operations: $Union(H_1,H_2)-O(n)$, $DecreaseKey(H,x,k)-O(log(n))$, 
$Delete(H,x)-O(log(n))$
% 3. lots of moving parts in this: here's a \href{https://www.youtube.com/watch?v=0vsX3ZQFREM}{6 minute video} on fibonacci heaps; 
% \href{https://www.utsc.utoronto.ca/~atafliovich/cscb63/content/week10/clrs_fibonacci_chapter.pdf}{chapter from CLRS}, 
% and \href{https://www.cs.princeton.edu/~wayne/teaching/fibonacci-heap.pdf}{easy to follow slides}.

% 
% problems on disjoint sets are in homework 5