% ---- generic definition ----
UnionFind for Disjoint Sets: given \textit{n} singletons: 
$S_i = {i}$ for $i = 1, ..., n$
% operations
1. \textit{union}($S_i$, $S_j$) return $S_i U S_j$, where $S_i$ and $S_j$ are disjoint; 
2. \textit{findSet(i)} gives the set of \textit{i}.
3. \textit{makeSet(i)} creates a singleton set with \textit{i}
% generic definition from above can be omitted
\\
%
% these two heuristics improve running time
Disjoint set forests: set is a rooted tree; each node contains a member; 
root acts as the set representative
1. union (by rank): rank is (upper bound) height of node 
(\# of edges from leaf to node); make root with smaller rank point to root with larger rank
2. findSet (by path compression): first pass finds the path to the root; 
second pass updates all the nodes in the path to point to root

% complexity
(worst-case) $T(n) = O(m\alpha(n))$, where m = total operations, n = \# of elements, 
$\alpha(n) (\le 4)$ is a very slowly growing fun (inverse Ackermann's fun)
% the "Properties about ranks:" bit can be skipped as it proves the time 
% complexity of UnionFind is slowly growing: O(mlog*n) or O(mα(n))

% including notes on two heaps for completeness - although I don't think we did
% problems on heaps or studied it in class?
% binary heaps
Binary Heaps: 
1. max heap property - parent is no smaller(or $\ge$) than its children;
2. as an array ($H[n]$) - for $H[i]$ , parent = $H[\lfloor i/2 \rfloor]$, 
left child = $H[2i]$, right child = $H[2i+1]$
3. heap sort (worst-case): $O(nlogn)$
4. heap height: $\Omega(logn)$
5. operations: $Insert(H, x)-log(n)$, $ExtractMax(H)-log(n)$, 
$FindMax(H)-O(1)$, $MakeHeap(H)-O(n)$
% useful to understand array ordering -- homework 2 problem 4
5. \href{https://www.mimuw.edu.pl/~erykk/algovis/heapsort.html}{Heap Sort Viz.}

% fibonacci heaps
Fibonacci Heaps/Mergeable Heaps: 
1. implemented using binary heaps
2. new operations: $Union(H_1,H_2)-O(n)$, $DecreaseKey(H,x,k)-O(log(n))$, 
$Delete(H,x)-O(log(n))$
3. lots of moving parts in this: here's a \href{https://www.youtube.com/watch?v=0vsX3ZQFREM}{6 minute video} on fibonacci heaps; 
\href{https://www.utsc.utoronto.ca/~atafliovich/cscb63/content/week10/clrs_fibonacci_chapter.pdf}{chapter from CLRS}, 
and \href{https://www.cs.princeton.edu/~wayne/teaching/fibonacci-heap.pdf}{easy to follow slides}.

% 
% problems on disjoint sets are in homework 5