\documentclass[8pt]{minimal}
\usepackage{graphicx} 
\usepackage{multicol}
\usepackage{lipsum} % For dummy text

% 0.1 inch margins for printing
\textwidth 8.3truein
\textheight 10.8truein
\oddsidemargin -0.9in
\topmargin -0.9in

\parindent 0pt
\def\baselinestretch{0.9}

\begin{document}
\begin{flushleft}
\begin{multicols}{3}


    \textbf{1. Mathematical Foundations}

    \textbf{1.1 Common Functions}

$e^x=1+x+{x^2\over 2!}+{x^3\over 3!}+\cdots$ ($e \approx 2.718$)

$\lim_{n\to\infty}{(1+{x\over n})}^n=e^x$.

$\log(ab)=\log a+\log b$, $\log ({a\over b})=\log a-\log b$.

$\log_a b={{\log_c b}\over {\log_c a}}$.

$\log_a b^n=n\log_a b\not={(\log_a b)}^n$, $a^{\log_a n}=n$,
$a^{\log_c b}=b^{\log_c a}$.

$ln(1+x)=x-{x^2\over 2}+{x^3\over 3}-{x^4\over 4}+\cdots$.

$n!=n\cdot (n-1)!$, $0!=1$.

Sterling's approximation: $n!=\sqrt{2\pi n}({n\over e})^n(1+
\Theta({1\over n}))$. (Note: $\Theta$ means having the same order of 
magnitude.) The following approximation also holds:
$n!=\sqrt{2\pi n}({n\over e})^ne^{\alpha_n}$, where
${1\over 12n+1}<\alpha_n<{1\over 12n}$.

$\log n!=\Theta(n\log n)$.

Log star function: how many times log must be applied before the result is $\le 1$:

$\log^* n=\min\{i\ge0:\log^{(i)}n\le1\}$

$\log^*2=1$, $\log^* 4=2$, $\log^*16=3$, $\log^*65536=4$, $\log^*2^{65536}=5$

    \textbf{1.2 Asymptotic Notation}

% Big vs. Little O explained: https://stackoverflow.com/questions/1364444/difference-between-big-o-and-little-o-notation

\item $f(n)=O(g(n))$ if $\exists c, n_0$ such that 
$f(n)\le cg(n)$ for $n\ge n_0$.

\item $f(n)=\Omega(g(n))$ if $\exists c, n_0$ such that
$f(n)\ge cg(n)$ for $n\ge n_0$.

\item $f(n)=\Theta(g(n))$ if $\exists c_1, c_2, n_0$ such that
$c_1g(n)\le f(n)\le c_2g(n)$ for $n\ge n_0$.

\item $f(n)=o(g(n))$ if $\forall c$ $\exists n_0$ such that
$f(n)<cg(n)$ for $n\ge n_0$. 

\item $f(n)=\omega(g(n))$ if $\forall c$ $\exists n_0$ such that 
$f(n)>cg(n)$ for $n\ge n_0$.


An alternative definition for $f(n)=o(g(n))$ is 
$\lim_{n\to\infty}{f(n)\over g(n)}=0$. Likewise, an alternative definition
for $f(n)=\omega(g(n))$ is $\lim_{n\to\infty}{f(n)\over g(n)}=\infty$.

{\em Rule of thumb:} constant $\le$ polylogarithmic $\le$ polynomial
$\le$ exponential $\le$ superexponential.

{\em Example:} $1$, $\sqrt{\log n}$, $\ln n$, $(\log n)^2$, $\sqrt n$,
$\sqrt n \log n$, $10n$, $n\log n$, $n^2$, $n^{\log\log n}$,
$2^n$, $n2^n$, $n!$, $2^{2^n}$.

Taking logarithms helps: $f(n)=O(g(n))$ iff $\log f(n)=O(\log g(n))$.


    \textbf{1.3 Summations/Series}

Arithmetic: $\sum_{i=1}^{n} i = {1\over2}n(n+1)$

Geometric: $\sum_{i=0}^{n} = {{r^{n+1}-1}\over{r-1}}$ for
$r\not=1$

\;\;\;\;\;\;\;\;\;\;\;\;\,\;\;\;\;\;\;\;\;\;\;\;\;\;\;\;\;$={1\over{1-r}}$ for $|r|<1$.

Harmonic: $1+{1\over2}+\cdots+{1\over n}
=\ln n+\gamma+{\epsilon\over{2n}}$

for $\gamma=0.5772156649\ldots$
(Euler's constant) and $0<\epsilon<1$.


Binomial: ${n\choose 0}+{n\choose 1}+{n\choose 2}+
\cdots+{n\choose n}=2^n$.

Other useful sums:

$\sum_{i=1}^n i^2={1\over6}n(n+1)(2n+1)$.

$\sum_{i=1}^n i^3=(\sum_{i=1}^n i)^2$.

$\sum_{i=1}^n ix^{i-1}={{nx^{n+1}-(n+1)x^n+1}\over{(x-1)^2}}$.


    \textbf{1.4 Proof Techniques}

Contradiction

Induction: basis, hypothesis, step



    \textbf{1.5 Solving Recurrences}

Iteration: Appy recurrence until can figure out summation

{\em Example:} $T(n)=3T({n\over4})+n$.

Assume $n=4^k$, so $k=\log_4 n$.
\begin{eqnarray*}
T(n)&=&3T({n\over4})+n\\
&=&3^2T({n\over{4^2}})+{3\over4}n+n\\
&=&3^3T({n\over{4^3}})+({3\over4})^2n+({3\over4})n+n\\
&=&\cdots\\
&=&3^kT({n\over{4^k}})+(({3\over4})^{k-1}+\cdots+({3\over4})+1)n\\
&=&3^{\log_4 n}+{{1-({3\over4})^{\log_4 n}}\over{1-{3\over4}}}n\\
&=&4n-3\cdot 3^{\log_4 n}\\
&=&O(n).
\end{eqnarray*}


Recursion Tree:

Master: does not cover all cases

{\em Master Theorem:} If $T(n)=aT({n\over b})+f(n)$ for $a\ge1$ and $b>1$, then

(a) if $f(n)=O(n^{(\log_ba)-\epsilon})$ for some $\epsilon>0$, then
$T(n)=\Theta(n^{\log_ba})$;

(b) if $f(n)=\Theta(n^{\log_ba})$, then
$T(n)=\Theta(n^{\log_ba}\log n)$;

(c) if $f(n)=\Omega(n^{(\log_ba)+\epsilon})$ for $\epsilon>0$ and if
$af({n\over b})\le cf(n)$ for $c<1$ and all large $n$, then
$T(n)=\Theta(f(n))$.


{\em Example:} $T(n)=3T({n\over4})+n\log n$.

$a=3$, $b=4$, and $f(n)=n\log n$.
Case (c) applies. So $T(n)=\Theta(n\log n)$.

Substitution:


    \textbf{2. Analysis of Algorithms}

    \textbf{2.1 Overview}

    \textbf{2.2 Worst-Case Analysis}

    \textbf{2.3 Average-Case Analysis}

    \textbf{2.4 Amortized Analysis}

    \textbf{2.5 Probabilistic Analysis}


    \textbf{3. Data Structures}

    ln3


    \textbf{4. Greedy Method}

    ln4


    \textbf{5. Divide and Conquer}

    ln5


    \textbf{6. Dynamic Programming}

    ln6


    \textbf{7. The Lower Bound Theory}

    ln7


    \textbf{8. Online Algorithms}

    ln8
    

    % The content will fill out the three columns without the lipsum
    % Get rid of them when done
    \lipsum
    \lipsum

\end{multicols}
\end{flushleft}
\end{document}